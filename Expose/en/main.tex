\documentclass[a4paper,11pt]{article}

% -------------------------
% Pakete
% -------------------------
\usepackage[T1]{fontenc}
\usepackage[utf8]{inputenc}
\usepackage[english]{babel}
\usepackage{microtype}
\usepackage{float}
\usepackage{geometry}
\geometry{a4paper, left=2.5cm, right=2.5cm, top=4cm, bottom=5cm, bindingoffset=1cm}

\usepackage{graphicx}
\usepackage{svg}
\svgsetup{inkscapelatex=true}
\usepackage{import}

\usepackage{xcolor}
\usepackage{setspace}
\onehalfspacing

\usepackage{tabularx}
\usepackage{multirow}
\usepackage{siunitx}
\usepackage{rotating}
\usepackage{comment}
\usepackage{fancyhdr}
\usepackage{marvosym}
\usepackage{listings}
\lstset{
  language=Python,
  basicstyle=\footnotesize\ttfamily,
  captionpos=b,
  frame=tb,
  commentstyle=\bfseries,
  stringstyle=\ttfamily,
  keywordstyle=\bfseries,
  breaklines=true,
  aboveskip=10mm,
  belowskip=10mm,
  showstringspaces=false,
  numbers=left,
  numberstyle=\tiny,
  numbersep=5pt
}

\usepackage[printonlyused,withpage]{acronym}
\usepackage{cite}
\usepackage{pdfpages}

% -------------------------
% PDF/A (pdfx lädt hyperref selbst – kein separates \usepackage{hyperref} nötig)
% -------------------------
\RequirePackage{filecontents}
\begin{filecontents*}{\jobname.xmpdata}
\Title{Document’s title}
\Author{Author’s name}
\Language{en-US}
\Subject{The abstract, or short description.}
\Keywords{keyword1\sep keyword2\sep keyword3}
\end{filecontents*}
\usepackage{colorprofiles}
\usepackage[a-1b,mathxmp]{pdfx}[2018/12/22]
\hypersetup{pdfstartview=} % hyperref ist durch pdfx aktiv

% -------------------------
% Eigene Kommandos / Konfiguration
% -------------------------
%
% own commands
%

%double empty page
\newcommand \myemptypage {
    \clearpage
    \thispagestyle{empty}
    \null
    \cleardoublepage
}

%create abstract environment that is not available in latex book style
\newcommand\abstractname{Abstract}
\newenvironment{abstract}{%
    \begin{center}%
        \normalfont\Large\bfseries \abstractname
    \end{center}%
    \it%
    }
    {}
      % deine Makros (benutzt u.a. \myemptypage, \Telefon, ...)

% Basis-Konfiguration (falls nötig; ansonsten später wieder aktivieren)
% % --- Basiskodierung und Fonts (optional, je nach Setup) ---
\usepackage[T1]{fontenc}
\usepackage[utf8]{inputenc}
\usepackage{lmodern}

% Palatino als Serif, Helvetica als Sans
\usepackage{palatino}
\usepackage{helvet}
\renewcommand{\sfdefault}{phv} % explizit Helvetica
\linespread{1.05}

% Mikrotypografie
\usepackage{microtype}

% Layout / Geometrie
\usepackage{geometry}
\geometry{a4paper, left=2.5cm, right=2.5cm, top=4cm,
          bottom=5cm, bindingoffset=1cm}

% Zeilenabstand
\usepackage{setspace}
\onehalfspacing

% Absätze
\usepackage{parskip}
\sloppy

% Überschriftenformatierung
\usepackage{titlesec}
\titleformat{\chapter}[display]
  {\LARGE\bfseries}{\chaptertitlename\ \thechapter}{10pt}{\huge\bfseries}

% Farben (für listings etc.)
\usepackage{xcolor}
\definecolor{darkblue}{rgb}{0.0,0.0,0.5}
\definecolor{grey}{rgb}{0.8,0.8,0.8}
\definecolor{lightgrey}{rgb}{0.95,0.95,0.95}

% Listings
\usepackage{listings}
\lstset{
  language=Python,
  basicstyle=\footnotesize\ttfamily,
  captionpos=b,
  frame=tb,
  commentstyle=\color{gray}\bfseries,
  stringstyle=\ttfamily,
  keywordstyle=\color{darkblue}\bfseries,
  breaklines=true,
  aboveskip=10mm,
  belowskip=10mm,
  showstringspaces=false, % ← hier war bei dir das Komma fehlend
  numbers=left,
  % stepnumber=5,
  numberstyle=\tiny,
  numbersep=5pt
}

% comment-Umgebung
\usepackage{comment}

% Bezeichnungen anpassen
\renewcommand{\bibname}{References}
\renewcommand{\lstlistlistingname}{List of Listings}

% Für \url und klickbare Links (falls du DOIs/URLs nutzt)
\usepackage{hyperref}



% Dokument-Variablen
\newcommand{\mytype}{Analytical Study of Automated Component Integrations in XNAT}
\newcommand{\mycourse}{Bachelor}
\newcommand{\myauthor}{Tanae Bousfiha}
\newcommand{\mydepartment}{HAWK Hochschule für angewandte wissenschaft und kunst \\Institut für Medizinische Informatik der Universitätsmedizin Göttingen}
\newcommand{\mysubmissiondate}{31. August 2025}
\newcommand{\supervisor}{Philip Zaschke, MSc.}

% -------------------------
% Start Dokument
% -------------------------
\begin{document}

% Vortitel / Römische Seitenzahlen
\pagenumbering{roman}
\setcounter{page}{1}

% Titel-/Coverpage (nutzt Variablen oben und ggf. SVG-Logos)
%
% title page
%

\begin{titlepage}
    %--- logo ---
    \normalsize
    \begin{tabularx}{\textwidth}{lXr}
        \multirow{2}{*}{\includegraphics[width=6.5cm]{images/goe-logo.jpg}} 
        & & \\
    \end{tabularx}

    %default settings for the rest
    \large
    \centering

    \vspace{3cm}

    \textbf{\LARGE \mytype}\\

    submitted in partial fulfillment of the\\
    requirements for the course ``\mycourse''

    \vspace{2cm}

    \textbf{\LARGE \ The Automation of implementation of Docker Container in XNAT}

    \vspace{2cm}

    \myauthor{ }

    \vspace{2cm}

    \mydepartment

    \vspace{2cm}

    Bachelor's and Master's Theses\\
    of the Center for Computational Sciences\\
    at the Georg-August-Universität Göttingen

    \vspace{0.2cm}

    \mysubmissiondate


    %--- new page ---
    \myemptypage
    \clearpage
    \thispagestyle{empty}
    \null
    \flushleft
    \onehalfspacing
    \normalsize

    \vspace{12cm}

    Georg-August-Universität Göttingen\\
    Institute of Computer Science\\[3ex]
    Goldschmidtstraße 7\\
    37077 Göttingen\\
    Germany\\[3ex]

    \begin{tabular}{@{}ll}
        \Telefon & +49 (551) 39-172000\\
        \fax & +49 (551) 39-14403\\
        \Letter & \href{mailto:office@informatik.uni-goettingen.de}{office@informatik.uni-goettingen.de}\\
        \Mundus & \url{www.informatik.uni-goettingen.de}\\
    \end{tabular}

    \vspace{1.0cm}

    \begin{tabular}{@{}ll}
        First Supervisor: & \myfirstsupervisor\\
        Second Supervisor:& \mysecondsupervisor\\
    \end{tabular}

    \clearpage
\end{titlepage}



% Abstract (falls du eine Datei einbindest, hier \input{...} einsetzen)
\clearpage\phantomsection\pdfbookmark{\abstractname}{abstract}
\thispagestyle{empty}
% \begin{abstract}
    The area of the automation of implementation of Docker Container in xnat (eXtensible Neuroimaging Archive Toolkit) is attracting considerable interest in the Somnolink Project. This paper provides an overview of the automation of implementation of Docker Container in xnat. It was hypothesized that a Container could be implemented in a Project in xnat and to rework all the files out and to reupload the result files back to their places. The approach is partly based on using a Python Script that worked with REST API and create a Dockerfile and askes the user for an external/ Container Script and extract all the files from the Open source Xnat and upon the conclusion of the experience, the Docker Container in xnat should be in ‘Complete’ turned and the result files should be uploaded. Experimental application of the methods demonstrated that the container could not receive the extracted files via REST API, which led to a not feasible neither processing nor uploading files. Results revealed a significant correlation between the workflow data of xnat and the Mounting of data in the Docker container. These findings demonstrate that the extraction of all the files from all the levels of xnat leads to a problem Mounting in the xnat host before arriving to the container. And the fact that the Container has no access to Data Bank of xnat conducts that the effectiveness of the method proved to be context dependent and thus limited.
\end{abstract}

 % optional

% Abkürzungen nach Abstract zurücksetzen
\acresetall

% Inhaltsverzeichnis
\clearpage\phantomsection\pdfbookmark{\contentsname}{toc}
\tableofcontents
\clearpage

% (Optional) Abbildungsverzeichnis – nur, wenn du wirklich Abbildungen mit \caption hast
% \listoffigures
% \myemptypage

% Arabische Nummerierung ab dem Hauptteil
\pagenumbering{arabic}
\setcounter{page}{1}

% -------------------------
% Hauptinhalt (NUR hier Kapitel/Sections einbinden!)
% -------------------------


\chapter{Introduction}

Artificial intelligence holds significant, largely untapped potential within healthcare. Successful implementation of \ac{AI} could alleviate workload pressures on healthcare professionals while simultaneously enhancing the quality of their work through error reduction and increased precision. Furthermore, AI could empower patients to take a more active role in managing their own health and contribute to a reduction in avoidable hospital admissions~\cite{aung_promise_2021}.


XNAT is an open-source platform for storing, managing, and analyzing imaging data and research labs around the world~\cite{xnat_implemetation.php}. Its flexible design makes it suitable for many different imaging projects~\cite{xnat_about}.


The open-source XNAT archiving system has expanded to include biosignal data and integrated machine learning, allowing AI models to transform it into an active analysis hub through automated processing~\cite{marcus_extensible_2007}.

In the Somnolink project, XNAT facilitates interoperable sleep data exchange to improve \ac{OSA} from diagnosis to therapy. AI systems integrated within XNAT support early patient identification, treatment planning, and therapy  adherence~\cite{SOMNOLINK}.

Studies on AI tools in clinical research often emphasize particular technical infrastructures, neglecting generalizable implementation strategies within data platforms. For instance, by combining NVIDIA’s Clara Deploy with a customized XNAT, Hawkins et al. (2023)~\cite{hawkins_implementation_2023} showed how machine-learning pipelines can be integrated into clinical workflows for near real-time processing. This approach transformed XNAT into an active hub for pipeline-driven analysis, with results stored for clinician review. Building on this, Satrajit Chakrabarty et al.~\cite{chakrabarty_deep_2023} presented an end-to-end framework capable of automating \ac{MRI} scan classification and tumor segmentation using AI-based methods in neuro-oncology studies.

Although these examples demonstrate technical feasibility, existing studies often emphasize specific infrastructures or disease context rather than examining broader, generalizable implementation strategies within data platforms. This thesis addresses that gap by investigating two possible approaches for integrating \ac{ML} within XNAT, with the goal of identifying best practices, recognizing limitations, and proposing future measures to support scalable and generalizable AI implementations in healthcare data platforms.

First, this bachelor thesis will examine the Docker Container approach within XNAT, considering its benefits, disadvantages, and future development. It will assess its relevance to the Somnolink project and discuss its applicability regarding sustainability and the FAIR principles. The findings will be synthesized into a clear User Guide for the practical application of the Docker Container approach.
Second, the study will analyze the XNAT Pipeline approach and compare it to the  assessment of the Docker Container method, exploring future measurement possibilities for both.


Chapter 2 lays the fundamentals by explaining the primary concepts necessary for understanding the thesis. Chapter 3 then details the Docker Container method, explaining its deployment within XNAT and introducing the automation script. A user guide is provided for users. Following this, the thesis addresses the Pipeline method and its deployment within XNAT. Chapter 4 provides a comprehensive analysis of both the Docker Container and Pipeline methods within XNAT. It examines their generalizability, relevance to the Somnolink project including sustainability and FAIR criteria, and their advantages and disadvantages. Furthermore, the chapter explores potential future measurements to further assess both methods.



 





\section{Fundamentals}

During an internship centered on the investigation and prototypical implementation of automated component integrations in XNAT, it was hypothesized that a Container could be implemented in a Project in XNAT to extract all the files and upload the result files back to their places. The approach was partly based on using a Python script that worked with the REST API, created a Dockerfile, and asked the user for an external Container Script to extract all the files from the open-source XNAT. Upon the conclusion of the experience, the Docker Container in XNAT should be marked as ‘Complete’ and the result files should be uploaded. 

Experimental application of the methods demonstrated that the container could not receive the extracted files via REST API, which made neither processing nor uploading files feasible. For another investigation, attention shifted to the traditional pipeline method. Consequently an investigation of this method remains inevitable, in order to figure out which methods is the most efficient.

  






\section{Research Questions}
The bachelor thesis builds upon the knowledge gained during the Bachelor Praxis Project.
 First, it will provide a comprehensive analysis of the Docker Container approach within XNAT. This includes identifying the advantages, limitations, and future development potential.
 The relevance of this method will also be demonstrated through its application in the Somnolink project, giving a particular attention to sustainability aspects such as the FAIR guiding principles~\cite{wilkinson_fair_2016}.

 As shown in Figure~\ref{fig:diagram-core-libraries}, the prototypical ML model execution in XNAT illustrates how fragmented data can be centralized and analyzed efficiently.

Second, a comparative analysis will be conducted with the alternative Pipeline. Focusing at first how the method is working and exposing the efficiently or the limitations and the negative or the positive point of this method, including the future development potential.  
Third, based on a insight gained from the two approaches, a User Guide is developed with details, with the aim of providing the user with clarity and an instruction manual.

\begin{figure}[H]
  \centering
  \def\svgwidth{0.8\linewidth}
  \input{expose1.pdf_tex}
  \caption{Schema: Illustrates the prototypical ML model execution }
  \label{fig:diagram-core-libraries}
\end{figure}


Finally, following on with the actions and the measurements to take in order to make the two methods more generalizable and more sustainable, including the potential development for a GUI-based approach.   


\begin{figure}[H]
  \centering
  \def\svgwidth{1.1\linewidth}
  \input{exposeplan.pdf_tex}
  \caption{Diagram: exposes the plan for the work packages}
\end{figure}

This following work packages have been defined in this period from 15.09.25 to 17.11.25.


% -------------------------
% Literatur
% -------------------------
\bibliographystyle{IEEEtran}
\bibliography{content/references}
\addcontentsline{toc}{section}{\bibname} % article: section (nicht chapter)
\clearpage

% (Optional) Anhang
% \appendix
% \section{Appendix A}
% ...

\end{document}



