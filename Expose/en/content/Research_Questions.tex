\section{Research Questions}
The bachelor thesis builds upon the knowledge gained during the Bachelor Praxis Project.
 First, it will provide a comprehensive analysis of the Docker Container approach within XNAT. This includes identifying the advantages, limitations, and future development potential.
 The relevance of this method will also be demonstrated through its application in the Somnolink project, giving a particular attention to sustainability aspects such as the FAIR guiding principles~\cite{wilkinson_fair_2016}.

 As shown in Figure~\ref{fig:diagram-core-libraries}, the prototypical ML model execution in XNAT illustrates how fragmented data can be centralized and analyzed efficiently.

Second, a comparative analysis will be conducted with the alternative Pipeline. Focusing at first how the method is working and exposing the efficiently or the limitations and the negative or the positive point of this method, including the future development potential.  
Third, based on a insight gained from the two approaches, a User Guide is developed with details, with the aim of providing the user with clarity and an instruction manual.

\begin{figure}[H]
  \centering
  \def\svgwidth{0.8\linewidth}
  \input{expose1.pdf_tex}
  \caption{Schema: Illustrates the prototypical ML model execution }
  \label{fig:diagram-core-libraries}
\end{figure}


Finally, following on with the actions and the measurements to take in order to make the two methods more generalizable and more sustainable, including the potential development for a GUI-based approach.   


\begin{figure}[H]
  \centering
  \def\svgwidth{1.1\linewidth}
  \input{exposeplan.pdf_tex}
  \caption{Diagram: exposes the plan for the work packages}
\end{figure}

This following work packages have been defined in this period from 15.09.25 to 17.11.25.
