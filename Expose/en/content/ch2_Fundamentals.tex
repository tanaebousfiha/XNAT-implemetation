


\section{Fundamentals}

During an internship centered on the investigation and prototypical implementation of automated component integrations in XNAT, it was hypothesized that a Container could be implemented in a Project in XNAT to extract all the files and upload the result files back to their places. The approach was partly based on using a Python script that worked with the REST API, created a Dockerfile, and asked the user for an external Container Script to extract all the files from the open-source XNAT. Upon the conclusion of the experience, the Docker Container in XNAT should be marked as ‘Complete’ and the result files should be uploaded. 

Experimental application of the methods demonstrated that the container could not receive the extracted files via REST API, which made neither processing nor uploading files feasible. For another investigation, attention shifted to the traditional pipeline method. Consequently an investigation of this method remains inevitable, in order to figure out which methods is the most efficient.

  





