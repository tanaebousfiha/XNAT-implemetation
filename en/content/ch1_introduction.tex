

\chapter{Introduction}

Artificial intelligence holds significant promises in healthcare. Successful implementation of \ac{AI} could ease clinician's workload, enhance diagnostic accuracy and enable the patient to take more control of their healthcare, thereby reducing preventable hospital admissions.~\cite{aung_promise_2021}. XNAT is an open-source platform for storing, managing, and analyzing imaging data~\cite{xnat_implemetation.php}. its archiving system has expanded to include biosignal data and integrated machine learning automated processing, transforming it into an active analysis hub~\cite{marcus_extensible_2007}. In the Somnolink project, XNAT facilitates interoperable sleep data exchange to improve \ac{OSA} from diagnosis to therapy. AI systems integrated within XNAT support early patient identification, treatment planning, and therapy  adherence~\cite{SOMNOLINK}.

Studies on AI tools in clinical research often emphasize particular technical infrastructures, neglecting generalizable implementation strategies within data platforms. For instance, by combining NVIDIA’s Clara Deploy with a customized XNAT, Hawkins et al. (2023)~\cite{hawkins_implementation_2023} showed how machine-learning pipelines can be integrated into clinical workflows for near real-time processing. This approach transformed XNAT into an active hub for pipeline-driven analysis, with results stored for clinician review. Building on this, Satrajit Chakrabarty et al.~\cite{chakrabarty_deep_2023} presented an end-to-end framework capable of automating \ac{MRI} scan classification and tumor segmentation using AI-based methods in neuro-oncology studies.
While AI integration is technically feasible, current examples often lack generalizable implementation strategies. This thesis addresses that gap by investigating two possible approaches for integrating \ac{ML} within XNAT, with the goal of identifying best practices, recognizing limitations, and proposing future measures to support scalable and generalizable AI implementations in healthcare data platforms. The study examines the Docker approach, including an automation script, its benefits, disadvantages, future development, and relevance to the Somnolink project, alongside a practical User Guide. Furthermore, the XNAT Pipeline approach was analyzed with same methodology as the Dockermethod.






 

