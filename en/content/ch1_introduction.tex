

\chapter{Introduction}

Artificial intelligence holds significant, largely untapped potential within healthcare. Successful implementation of \ac{AI} could alleviate workload pressures on healthcare professionals while simultaneously enhancing the quality of their work through error reduction and increased precision. Furthermore, AI could empower patients to take a more active role in managing their own health and contribute to a reduction in avoidable hospital admissions~\cite{aung_promise_2021}.


XNAT, developed at Washington University’s Neuroinformatics Research Group, is an open-source platform for storing, managing, and analyzing imaging data. Its flexible design makes it suitable for many different imaging projects\footnote{\url{https://xnat.org/about/} (last accessed: 24.09.2025)}.


The open-source XNAT archiving system has expanded to include biosignal data and integrated machine learning, allowing AI models to transform it into an active analysis hub through automated processing~\cite{marcus_extensible_2007}.

In the Somnolink project, XNAT facilitates interoperable sleep data exchange to improve \ac{OSA} from diagnosis to therapy. AI systems integrated within XNAT support early patient identification, treatment planning, and therapy  adherence~\cite{internetredaktion_somnolink_nodate}.

Studies on AI tools in clinical research often emphasize particular technical infrastructures, neglecting generalizable implementation strategies within data platforms. For instance, by combining NVIDIA’s Clara Deploy with a customized XNAT, Hawkins et al. (2023)~\cite{hawkins_implementation_2023} showed how machine-learning pipelines can be integrated into clinical workflows for near real-time processing. This approach transformed XNAT into an active hub for pipeline-driven analysis, with results stored for clinician review. Building on this, Satrajit Chakrabarty et al. ~\cite{chakrabarty_deep_2023} presented an end-to-end framework capable of automating \ac{MRI} scan classification and tumor segmentation using AI-based methods in neuro-oncology studies.

Although these examples demonstrate technical feasibility, existing studies often emphasize specific infrastructures or disease context rather than examining broader, generalizable implementation strategies within data platforms. This thesis addresses that gap by investigating the methods for integrating \ac{ML} within XNAT, with the goal of identifying best practices, recognizing limitations, and proposing future measures to support scalable and generalizable AI implementations in healthcare data platforms.

This bachelor thesis will examine the Docker Container approach within XNAT, considering its benefits, disadvantages, and future development. It will assess its relevance to the Somnolink project, particularly regarding sustainability and FAIR principles. The findings will be synthesized into a clear User Guide for the practical application of the Docker Container approach.
The study will also analyze the XNAT Pipeline approach and compare it to the Docker Container method, exploring future measurement possibilities for both.

Chapter 2 lays the fundamentals by explaining the primary concepts necessary for understanding the thesis. Chapter 3 then details the Docker Container method, explaining its deployment within XNAT and introducing the automation script. A user guide is provided for users. Following this, the thesis addresses the Pipeline method and its deployment within XNAT.

In Chapter 4 provides a comprehensive analysis of both the Docker Container and Pipeline methods within XNAT. It examines their  generalizability, relevance to the Somnolink project including sustainability and FAIR criteria, and their advantages and disadvantages. Furthermore the chapter explores potential future measurements to further assess both methods.



 

