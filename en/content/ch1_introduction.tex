

\chapter{Introduction}

\ac{AI} holds significant promises in healthcare. Successful implementation of artificial intelligence could ease clinician's workload, enhance diagnostic accuracy and enable the patient to take more control of their healthcare, thereby reducing preventable hospital admissions~\cite{aung_promise_2021}. XNAT is an open-source platform for storing, managing, and analyzing imaging data~\cite{xnat_about}. Its facilitates quality control processes and ensures secure data access and storage~\cite{marcus_extensible_2007}. In the Somnolink project, XNAT stored interoperable sleep data exchange to improve \ac{OSA} from diagnosis to therapy~\cite{krefting_somnolink_2025}.
Studies on AI tools in clinical research often emphasize particular technical infrastructures, neglecting generalizable implementation strategies within data platforms. For instance, by combining NVIDIA’s Clara Deploy with a customized XNAT, Hawkins et al. (2023)~\cite{hawkins_implementation_2023} showed how machine-learning pipelines can be integrated into clinical workflows. This approach transformed XNAT into an active hub for pipeline analysis, with results and clinician feedback stored in XNAT. Building on this, Satrajit Chakrabarty et al.~\cite{chakrabarty_deep_2023} presented an end-to-end framework capable of automating \ac{MRI} scan classification and tumor segmentation using AI-based methods in neuro-oncology studies. 

While AI integration is technically feasible, current examples often lack generalizable implementation strategies. 
This thesis addresses that gap by investigating two possible approaches for integrating \ac{ML} within XNAT. The following research questions are posed: how can the Docker container method be automated in XNAT? can the pipeline method also be automated? how can the pipeline approach be effectively integrated within XNAT? what are the strengths, limitations, and sustainability implications of both approaches in relation to the Somnolink project and the FAIR principles? what measurements can be applied to improve and optimize both methods for future implementations?






 
