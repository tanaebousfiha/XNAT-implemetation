% --- Basiskodierung und Fonts (optional, je nach Setup) ---
\usepackage[T1]{fontenc}
\usepackage[utf8]{inputenc}
\usepackage{lmodern}

% Palatino als Serif, Helvetica als Sans
\usepackage{palatino}
\usepackage{helvet}
\renewcommand{\sfdefault}{phv} % explizit Helvetica
\linespread{1.05}

% Mikrotypografie
\usepackage{microtype}

% Layout / Geometrie
\usepackage{geometry}
\geometry{a4paper, left=2.5cm, right=2.5cm, top=4cm,
          bottom=5cm, bindingoffset=1cm}

% Zeilenabstand
\usepackage{setspace}
\onehalfspacing

% Absätze
\usepackage{parskip}
\sloppy

% Überschriftenformatierung
\usepackage{titlesec}
\titleformat{\chapter}[display]
  {\LARGE\bfseries}{\chaptertitlename\ \thechapter}{10pt}{\huge\bfseries}

% Farben (für listings etc.)
\usepackage{xcolor}
\definecolor{darkblue}{rgb}{0.0,0.0,0.5}
\definecolor{grey}{rgb}{0.8,0.8,0.8}
\definecolor{lightgrey}{rgb}{0.95,0.95,0.95}

% Listings
\usepackage{listings}
\lstset{
  language=Python,
  basicstyle=\footnotesize\ttfamily,
  captionpos=b,
  frame=tb,
  commentstyle=\color{gray}\bfseries,
  stringstyle=\ttfamily,
  keywordstyle=\color{darkblue}\bfseries,
  breaklines=true,
  aboveskip=10mm,
  belowskip=10mm,
  showstringspaces=false, % ← hier war bei dir das Komma fehlend
  numbers=left,
  % stepnumber=5,
  numberstyle=\tiny,
  numbersep=5pt
}

% comment-Umgebung
\usepackage{comment}

% Bezeichnungen anpassen
\renewcommand{\bibname}{References}
\renewcommand{\lstlistlistingname}{List of Listings}

% Für \url und klickbare Links (falls du DOIs/URLs nutzt)
\usepackage{hyperref}

