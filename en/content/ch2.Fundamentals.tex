
\chapter{Fundamentals}

An overview of the system and methods used is provided in this chapter, serving as a foundation for the more detailed examination of the approaches that follows. 

\section{The Extensible Neuroimaging Archive Toolkit}
The Extensible Neuroimaging Archive Toolkit (XNAT)~\cite{marcus_extensible_2007} is an open-source imaging informatics platform developed by the Neuroinformatics Research Group at Washington University. It facilitates common management, productivity, and quality assurance tasks for imaging and associated data. Thanks to its extensibility, XNAT can be used to support a wide range of imaging-based projects.\footnote{\url{https://www.xnat.org/about/} (accessed on 10 August 2025)} Due to the fact that XNAT offers the opportunity to use the power of high-performance computing on your data~\cite{zaschke_extending_2024}, 
pipeline processing is one of the options that can be done on XNAT, as well as the deployment of Docker containers.
\\
Additionally, XNAT has a structured website that is composed of a Dashboard that gives an overview of the site, a Project list that browses or searches available projects, and a Project page where all subjects and their data in this project are located. Along with Subject/Session pages where all details are given, including imaging sessions and scan-level details, as well as the section of scans where all the patient’s scans are found. 
This structure of XNAT provides the information on where the patient files are stored, including the composed URL for REST API use. A REST API is a type of application programming interface that adheres to the principles of the REST architecture. REST, which stands for Representational State Transfer, defines a set of conventions for designing web APIs in a consistent and scalable way.\footnote{\url{https://www.redhat.com/en/topics/api/what-is-a-rest-api} (accessed on 10 August 2025)}
\\
\section{The Container}
A container is a standard unit of software that packages up code and all its dependencies so the application runs quickly and reliably from one computing environment to another.\footnote{\url{https://www.docker.com/resources/what-container/} (accessed on 10 August 2025)}
Docker is a platform that allows certain applications to run in software containers. It was launched in 2013.\footnote{\url{https://fr.wikipedia.org/wiki/Docker\_(logiciel)} (accessed on 10 August 2025)} A Docker container allows the user to deploy any application in any environment without worrying about incompatibility issues, regardless of the machine’s configuration settings.\footnote{\url{https://sematext.com/glossary/docker/} (accessed on 10 August 2025)} Although there are some associated terms that will always be needed when it comes to using or learning about Docker containers.
One of the most important of them is the Dockerfile. The Dockerfile is a file that contains the build instructions for the image that the Docker Engine (an open-source containerization technology for building and containerizing your applications\footnote{\url{https://docs.docker.com/engine/} (accessed on 10 August 2025)}) will run. Conversely, there are numerous commands that must be written in the Dockerfile in order to create the image. After building the image, we can tag and push the image to Docker Hub. For this purpose, we just have to run some basic simple commands on a terminal.
\\
Another element to consider is the REST APIs. A REST API is an application programming interface (API) that follows the design principles of the REST architectural style. REST is short for Representational State Transfer and is a set of rules and guidelines about how you should build a web API.\footnote{\url{https://www.redhat.com/en/topics/api/what-is-a-rest-api} (accessed on 10 August 2025)} Luckily, XNAT provides its users with a list of REST APIs that make the exchange between the user and the website possible. 

\section{The JSON Command}
At the same time, if we want XNAT to execute my created Docker image and run a command, a \ac{JSON} command is needed. The JSON command is a collection of properties that describe the Docker image, which allows XNAT to understand what the command is about. In other words, the JSON command is a sort of configuration for XNAT. It answers the following questions for XNAT: What kind of image is it? Does it have a human-friendly name we can use for it? What does the command-line string look like? Does it need files? Where should they go? How do you want to get those out of XNAT? Does it produce any files at the end? Those have to get back into XNAT, so where should they go?\footnote{\url{https://wiki.xnat.org/container-service/json-command-definition} (accessed on 10 August 2025)} These are the fundamental knowledge elements needed in order to achieve automation of the process.

\section{The Pipeline}
Pipelines in XNAT are efficient lightweight applications that can be run on the project data, to facilitate sophisticated processing tasks or leverage the computational power of extensive computing clusters. Some pipeline-enabled workflow run automatically, such as auto-QC of images. And some of them require manual steps, like defining a region of interest. After integrating the pipeline in a project data, we can set up project-based workflows with project specific and experiment specific parameters, track a pipeline and send email notifications, and capture provenance information as the pipeline executes.\footnote{https://wiki.xnat.org/documentation/running-pipelines-in-xnat{last acceed 07.09.2025}}. The pipeline are used for tasks like converting DICOM to NIFTI, performing image quality assessment (QA), correcting MRI bias fields, and running advanced analysis such as the FreeSurfer pipeline.\footnote{https://github.com/jhuguetn/xnat-pipelines{last acced 07.09.2025}}.

\section{The Pipeline Descriptor}

Pipelines are Identified by using a pipeline descriptor and resource descriptors. A resource descriptor describes an executable. An executable is identified by its name, its location and the arguments that it takes.\footnote{https://wiki.xnat.org/documentation/xnat-pipeline-development-schema{last acceed 07.09.2025}}, in other word the Pipeline Descriptor is a recipe of the pipeline that tells XNAT exactly what a pipeline is, what is the input needed and where the output should be located, ensuring a seamless integration in XNAT.







