\chapter{User Guide}
To better understand the system created and mostly the motivation behind this automation project. We have inserted a table to list how the the Workflow is working and what are the requirement that the user needs to have in order to run the script.

\begin{table}[htbp]
  \centering
  \caption{System Requirements and Script Details for Automatic Containerization}
  \label{tab:automatic_script}
  \begin{tabular}{|l|p{9cm}|}
  \hline
  \textbf{System Requirements} & 
  - Python installed \newline
  - \texttt{requirements.txt} in the same folder as \texttt{automat\_f\_2.py} \newline
  - XNAT server accessible \newline
  - Docker Engine installed and running \newline
  - Logged in to Docker Hub \\  \hline
  \textbf{Steps to Follow} & 
  1. Run the \texttt{automat\_f\_2.py} script \newline
  2. Enter the prompted inputs \newline
  3. Wait until the container is launched \\
  \hline
  \textbf{What Happens Automatically} & 
  - Dockerfile is generated \newline
  - Docker image is built, tagged, and pushed \newline
  - Command JSON is generated \newline
  - Command is uploaded to XNAT \newline
  - Wrapper is activated site-wide and project-wide \newline
  - Container is launched via REST API \\
  \hline
  \textbf{Problems / Troubleshooting} & 
  - Push failed: check Docker Hub login \newline
  - Wrapper not found or JSON upload failed: check chosen command naming \newline
  - Command already exists: delete old command manually \newline
  - Container problem uploading files: try running \texttt{ls -al \*} on your output directory and check the container \texttt{stdout} log \\
  \hline
  \end{tabular}
\end{table}