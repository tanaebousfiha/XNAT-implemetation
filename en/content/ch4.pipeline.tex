\chapter{The Analysis of the Second Method: The Pipeline}
\section{XNAT ML Plugin and The Pipeline}

For the purpose of investigating and examining the prototypical implementation of automated component integrations in XNAT, it is relevant to cite the XNAT ML Plugin\footnote{\url{https://wiki.xnat.org/ml/} accessed on 10 August 2025}, which was previously an implementation method developed to incorporate machine learning workflows directly into XNAT. It offered capabilities for uploading, configuring, and training AI models. However, this approach was highly interdependent, and the XNAT team officially deprecated the ML Plugin and the associated Datasets Plugin. They are no longer maintained or recommended for production use. Another method that can be cited is the Pipeline method.

Pipeline is a method of implementation for automated data processing in XNAT, using \ac{XML} descriptors and server-side scripts/tools. In XNAT, a pipeline is used to automate data processing workflows on imaging and related data stored in the system. Pipelines allow researchers and developers to define and execute repeatable, structured sequences of tasks such as data conversion, quality control, analysis, and post-processing directly on XNAT-managed data, without manual intervention.

After analyzing the Pipeline implementation method, it was revealed that the architecture exhibits several limitations regarding scalability and flexibility, particularly when integrating new automated components. These pipelines require direct access to the server via SSH in order to manually copy the XML descriptor and associated scripts to a specific directory (\texttt{/data/xnat/pipeline/pipelines/}). This approach is not feasible in secure or restricted settings where developers or researchers do not have shell access to the production server.

Furthermore, following the analysis of the REST API endpoints that XNAT supports, it became apparent that XNAT offers only limited support for this feature. The only supported endpoint in the list is \texttt{POST /pipelines/launch/\{pipelineName\}}, which allows programmatic launching of an already registered pipeline in XNAT. However, for full automation, additional endpoints would be needed, such as one to enable a pipeline globally in XNAT and another to assign or activate a pipeline for a specific project. These functionalities are currently not exposed via the REST API, which means that while the execution of pipelines can be automated, the full deployment and project-level configuration still require manual steps.

In conclusion, a fully automated pipeline workflow using only a Python script is not currently possible. Only the launching of existing, pre-registered pipelines can be automated via REST.