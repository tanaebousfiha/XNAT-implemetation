\chapter{The Analysis of the Second Method: The Pipeline}
\section{Pipeline Installation and Configuration in XNAT}

Pipeline is a method of implementation for automated data processing in XNAT, using \ac{XML} descriptors and server-side scripts/tools. 
To implement a pipeline in XNAT, an XML descriptor is required, this descriptor serves as a connector between XNAT and the user script including definitions for metadata, execution steps, conditions, iterations, and outputs—and resource descriptors that specify how scripts or executable are launched with necessary parameters.
During the Pipeline analysis we initially focused on analyzing the implementation process of a OSA prediction based script with the Pipeline method. The Pipeline system requires two components: Pipeline Engine and the XNAT Pipeline Engine Plugin. By default, pipelines are located at /data/home/xnat/pipeline\footnote{\url{https://wiki.xnat.org/documentation/xnat-pipeline-development-schema} last Accessed: 10 September 2025).}   

\begin{lstlisting}[language=XML, caption={Example of a pipeline resource descriptor in XNAT}, label={lst:pipeline-xml}]
<?xml version="1.0" encoding="UTF-8"?>
<pip:Resource xmlns:pip="http://nrg.wustl.edu/pipeline">
    <pip:name>run_test_pipeline.py</pip:name>
    <pip:location>/usr/local/python_scripts</pip:location>
    <pip:commandPrefix>python</pip:commandPrefix>
    <pip:type>Executable</pip:type>
    <pip:description>Generates SOMETHING HERE</pip:description>
    <pip:input>
        <pip:argument id="configfilepath">
            <pip:description>Path to param file</pip:description>
        </pip:argument>
        <pip:argument id="host">
            <pip:description>XNAT Host</pip:description>
        </pip:argument>
        <pip:argument id="user">
            <pip:description>XNAT User</pip:description>
        </pip:argument>
        <pip:argument id="passwd" isSensitive="true">
            <pip:description>XNAT User password</pip:description>
        </pip:argument>
        <pip:argument id="pipelineLogFile">
            <pip:description>Path to pipeline log file</pip:description>
        </pip:argument>
    </pip:input>
</pip:Resource>
\end{lstlisting}
The resource descriptor XML for an XNAT pipeline specifies the executable's invocation details and argument handling—such as the location, command prefix, and any argument marked as `isSensitive` to hide credentials.\footnote{Mohana Ramaratnam (2013, July 25). “Re: Running pipeline with parameters from an XML?”, *xnat\_discussion* Google Group. Available at: \url{https://groups.google.com/g/xnat_discussion/c/ewhjL7upJf4} (last accessed 10 September 2025).}
The  Listing~\ref{lst:pipeline-xml} presents a sample resource descriptor for a Python script, specifying its location, command prefix, and input arguments such as configuration file, XNAT host, user credentials (with password marked as sensitive), and log file path. This example illustrates how the schema integrates execution details with security considerations for sensitive data.
The commands shown are part of the process of setting up a classic XML-based pipeline in XNAT.
The first command copies the file my\_pipeline.xml, which contains the XML-based definition of the pipeline
workflow, into the directory /data/xnat/pipeline/pipelines/my\_pipeline/, where XNAT expects all pipeline
descriptors to be stored.
The second command copies the file run.sh—a shell script that performs the actual data processing defined in the
pipeline into the same directory. This ensures that both the pipeline definition and the script it executes are
located together in the correct structure required by the XNAT Pipeline Engine. Finally, the third command
makes the run.sh script executable using chmod +x, which is essential because XNAT will later attempt to run
this script as a command. Without executable permissions, XNAT would not be able to start the script, and the pipeline execution would fail.
\footnote{XNAT Documentation (2025): \url{https://wiki.xnat.org/documentation/installing-pipelines-in-xnat} last Accessed 10 September 2025}

\section{Pipeline Development and Integration in XNAT}
On the contrary, building and enabling a pipeline in XNAT involves considerably more manual steps compared to the Docker-based approach.
First we have to enter the path to the pipeline in XNAT under Admin>Pipelines>Register new Pipelines> And then, in the appropriate field, we enter the path. Then we have to build the option run the Pipeline.In order for users to execute the Pipeline from the XNAT interface an administrator must configure the relevant data type to include a pipeline launch action. This involves editing the data type settings under Administer >Data Types and adding a new action as PipelineScreen\_launch\_pipeline—in the Available Report Actions section, with a suitable display label like Build or Run Pipeline.\footnote{\url{https://wiki.xnat.org/documentation/installing-pipelines-in-xnat}last accessed 10.09.2025}

\section{The Pipeline and the REST API}
With the aim of analyzing the pipeline method more deeply, it is worth noting that XNAT provides several REST API endpoints for managing and running pipelines the most relevant endpoint for automation is \url{POST /pipelines/launch/{pipelineName}} \footnote{\url{://wiki.xnat.org/xnat-api/xnat-pipeline-api} last Accessed 10.09.2025}
This allows us to programmatically launch an already registered pipeline in XNAT. However, for full
automation, additional endpoints would be needed, such as one to enable a pipeline globally in XNAT and
another to assign or activate a pipeline for a specific project. These functionalities are currently not exposed via
the REST API, which means that while the execution of pipelines can be automated, the full deployment and
project-level configuration still require manual steps.

\section{Evaluating Pipeline and Docker Approaches in XNAT}
\section{Discussion and Conclusion and future measurements}


