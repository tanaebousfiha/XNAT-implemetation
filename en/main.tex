% !TeX spellcheck = fr_FR
\PassOptionsToPackage{hidelinks}{hyperref}

\documentclass[10pt,a4paper,openany]{book} % openany: Kapitel starten auf beliebiger Seite

% --- Sprache & Encoding ---
\usepackage[T1]{fontenc}
\usepackage[utf8]{inputenc}
\usepackage[english]{babel}

% === SCHRIFTEN (sichtbar anderes Design) ======================
\usepackage{tgtermes}        % Serif (Text) Times-ähnlich
\usepackage{newtxmath}       % Mathe passend zu Times
\usepackage[scaled=0.92]{helvet} % Sans-Serif (Überschriften)
\usepackage{inconsolata}     % Monospace (Code)
% ===============================================================

% --- Seitenränder kompakt ---
\usepackage[a4paper,margin=1.8cm]{geometry}

% --- Grundpakete ---
\usepackage{graphicx}
\usepackage{svg}
\svgsetup{inkscapelatex=true}
\usepackage{pdfpages}
\usepackage{booktabs}
\usepackage{array}
\usepackage{multirow}
\usepackage{ltablex}\keepXColumns
\usepackage{siunitx}
\usepackage[printonlyused,withpage]{acronym}
\usepackage[numbers]{natbib}
\usepackage{fancyhdr}
\usepackage{comment}
\usepackage{rotating}
\usepackage{import}
\usepackage{pythontex}
\usepackage{adjustbox}
\usepackage{float}
\usepackage{tikz}
\usepackage{hyperref}

% --- Listings: farbiger Code ---
\usepackage{xcolor}
\usepackage{listings}

\definecolor{codegreen}{rgb}{0,0.6,0}
\definecolor{codegray}{rgb}{0.5,0.5,0.5}
\definecolor{codepurple}{rgb}{0.58,0,0.82}
\definecolor{backcolour}{rgb}{0.95,0.95,0.92}

\lstdefinestyle{mystyle}{
  backgroundcolor=\color{backcolour},
  commentstyle=\color{codegreen},
  keywordstyle=\color{magenta},
  numberstyle=\tiny\color{codegray},
  stringstyle=\color{codepurple},
  basicstyle=\ttfamily\footnotesize,  % nutzt Inconsolata
  breaklines=true,
  numbers=left,
  numbersep=5pt,
  showstringspaces=false,
  tabsize=2,
  frame=single,
  captionpos=b
}
\lstset{style=mystyle}

% --- Überschriften in Sans-Serif (Helvetica, fett) ---
\usepackage{sectsty}
\allsectionsfont{\sffamily\bfseries}

% ===== keine automatisch eingefügten Leerseiten =====
\let\cleardoublepage\clearpage
% ====================================================

% --- Deine eigenen Kommandos & Config ---
%
% own commands
%

%double empty page
\newcommand \myemptypage {
    \clearpage
    \thispagestyle{empty}
    \null
    \cleardoublepage
}

%create abstract environment that is not available in latex book style
\newcommand\abstractname{Abstract}
\newenvironment{abstract}{%
    \begin{center}%
        \normalfont\Large\bfseries \abstractname
    \end{center}%
    \it%
    }
    {}

% --- Basiskodierung und Fonts (optional, je nach Setup) ---
\usepackage[T1]{fontenc}
\usepackage[utf8]{inputenc}
\usepackage{lmodern}

% Palatino als Serif, Helvetica als Sans
\usepackage{palatino}
\usepackage{helvet}
\renewcommand{\sfdefault}{phv} % explizit Helvetica
\linespread{1.05}

% Mikrotypografie
\usepackage{microtype}

% Layout / Geometrie
\usepackage{geometry}
\geometry{a4paper, left=2.5cm, right=2.5cm, top=4cm,
          bottom=5cm, bindingoffset=1cm}

% Zeilenabstand
\usepackage{setspace}
\onehalfspacing

% Absätze
\usepackage{parskip}
\sloppy

% Überschriftenformatierung
\usepackage{titlesec}
\titleformat{\chapter}[display]
  {\LARGE\bfseries}{\chaptertitlename\ \thechapter}{10pt}{\huge\bfseries}

% Farben (für listings etc.)
\usepackage{xcolor}
\definecolor{darkblue}{rgb}{0.0,0.0,0.5}
\definecolor{grey}{rgb}{0.8,0.8,0.8}
\definecolor{lightgrey}{rgb}{0.95,0.95,0.95}

% Listings
\usepackage{listings}
\lstset{
  language=Python,
  basicstyle=\footnotesize\ttfamily,
  captionpos=b,
  frame=tb,
  commentstyle=\color{gray}\bfseries,
  stringstyle=\ttfamily,
  keywordstyle=\color{darkblue}\bfseries,
  breaklines=true,
  aboveskip=10mm,
  belowskip=10mm,
  showstringspaces=false, % ← hier war bei dir das Komma fehlend
  numbers=left,
  % stepnumber=5,
  numberstyle=\tiny,
  numbersep=5pt
}

% comment-Umgebung
\usepackage{comment}

% Bezeichnungen anpassen
\renewcommand{\bibname}{References}
\renewcommand{\lstlistlistingname}{List of Listings}

% Für \url und klickbare Links (falls du DOIs/URLs nutzt)
\usepackage{hyperref}




% --- Metadaten ---
\newcommand{\mytype}{Investigation and Prototypical Implementation of Automated Component Integrations in XNAT}
\newcommand{\mycourse}{Bachelor}
\newcommand{\myauthor}{Tanae Bousfiha}
\newcommand{\mydepartment}{HAWK Hochschule für angewandte wissenschaft und kunst 
\\Institut für Medizinische Informatik der Universitätsmedizin Göttingen}
\newcommand{\mysubmissiondate}{xx. August 2025}
\newcommand{\myfirstsupervisor}{Prof.Dr. Roman Grothausmann}

\begin{document}

  \pagenumbering{roman}
  \setcounter{page}{1}

  %--- cover page ---
  %
% title page
%

\begin{titlepage}
    %--- logo ---
    \normalsize
    \begin{tabularx}{\textwidth}{lXr}
        \multirow{2}{*}{\includegraphics[width=6.5cm]{images/goe-logo.jpg}} 
        & & \\
    \end{tabularx}

    %default settings for the rest
    \large
    \centering

    \vspace{3cm}

    \textbf{\LARGE \mytype}\\

    submitted in partial fulfillment of the\\
    requirements for the course ``\mycourse''

    \vspace{2cm}

    \textbf{\LARGE \ The Automation of implementation of Docker Container in XNAT}

    \vspace{2cm}

    \myauthor{ }

    \vspace{2cm}

    \mydepartment

    \vspace{2cm}

    Bachelor's and Master's Theses\\
    of the Center for Computational Sciences\\
    at the Georg-August-Universität Göttingen

    \vspace{0.2cm}

    \mysubmissiondate


    %--- new page ---
    \myemptypage
    \clearpage
    \thispagestyle{empty}
    \null
    \flushleft
    \onehalfspacing
    \normalsize

    \vspace{12cm}

    Georg-August-Universität Göttingen\\
    Institute of Computer Science\\[3ex]
    Goldschmidtstraße 7\\
    37077 Göttingen\\
    Germany\\[3ex]

    \begin{tabular}{@{}ll}
        \Telefon & +49 (551) 39-172000\\
        \fax & +49 (551) 39-14403\\
        \Letter & \href{mailto:office@informatik.uni-goettingen.de}{office@informatik.uni-goettingen.de}\\
        \Mundus & \url{www.informatik.uni-goettingen.de}\\
    \end{tabular}

    \vspace{1.0cm}

    \begin{tabular}{@{}ll}
        First Supervisor: & \myfirstsupervisor\\
        Second Supervisor:& \mysecondsupervisor\\
    \end{tabular}

    \clearpage
\end{titlepage}


  
  %--- statement page ---
  
\clearpage
\includepdf[pages=-,fitpaper=true]{ST1.pdf}
\clearpage



  %--- abstract ---
  \clearpage\phantomsection\pdfbookmark{\abstractname}{abstract}
  \thispagestyle{empty}
  %
\begin{abstract}
This work investigates existing methods for implementing workflows in the \ac{XNAT}. Building on this analysis hypothesis posits that, can efficiently process. The proposed solution employs a Python script that interacts with the \ac{REST API} to generate a Dockerfile, prompt users for an external container script, and extract files from the open-source XNAT system. The envisioned workflow achieves a ‘Complete’ state in XNAT upon finalization and uploads the processed result files accordingly.


Empirical testing demonstrated, however, that extracting files via the REST API was unsuccessful, rendering direct file processing and uploading unfeasible. Analysis of the results revealed a significant interdependence between XNAT’s internal data workflow and the Docker container’s ability to mount and access data. Specifically, comprehensive extraction attempts led to issues with data mounting on the XNAT host, preventing access by the Docker container. Consequently, since the container could not access XNAT’s data repositories directly, the practicality of the proposed approach proved to be context-dependent and ultimately limited in effectiveness. These findings highlight crucial technical constraints in integrating containerized processing workflows within XNAT and underscore the need for improved data access strategies in future developments.
\end{abstract}



  % reset acronyms after abstract
  \acresetall

  %--- table of contents ---
  \clearpage\phantomsection\pdfbookmark{\contentsname}{toc}
  \tableofcontents



  %--- list of acronyms ---
  \chapter*{List of Abbreviations}
\usepackage{graphicx}
\begin{acronym}[myacronyms]
    \acro{FYI}{For Your Information}    
\end{acronym}


  % arabic page numbers
  \pagenumbering{arabic}
  \setcounter{page}{1}

  %--- content ---
  

\chapter{Introduction}

Artificial intelligence holds significant, largely untapped potential within healthcare. Successful implementation of \ac{AI} could alleviate workload pressures on healthcare professionals while simultaneously enhancing the quality of their work through error reduction and increased precision. Furthermore, AI could empower patients to take a more active role in managing their own health and contribute to a reduction in avoidable hospital admissions~\cite{aung_promise_2021}.


XNAT is an open-source platform for storing, managing, and analyzing imaging data and research labs around the world~\cite{xnat_implemetation.php}. Its flexible design makes it suitable for many different imaging projects~\cite{xnat_about}.


The open-source XNAT archiving system has expanded to include biosignal data and integrated machine learning, allowing AI models to transform it into an active analysis hub through automated processing~\cite{marcus_extensible_2007}.

In the Somnolink project, XNAT facilitates interoperable sleep data exchange to improve \ac{OSA} from diagnosis to therapy. AI systems integrated within XNAT support early patient identification, treatment planning, and therapy  adherence~\cite{SOMNOLINK}.

Studies on AI tools in clinical research often emphasize particular technical infrastructures, neglecting generalizable implementation strategies within data platforms. For instance, by combining NVIDIA’s Clara Deploy with a customized XNAT, Hawkins et al. (2023)~\cite{hawkins_implementation_2023} showed how machine-learning pipelines can be integrated into clinical workflows for near real-time processing. This approach transformed XNAT into an active hub for pipeline-driven analysis, with results stored for clinician review. Building on this, Satrajit Chakrabarty et al.~\cite{chakrabarty_deep_2023} presented an end-to-end framework capable of automating \ac{MRI} scan classification and tumor segmentation using AI-based methods in neuro-oncology studies.

Although these examples demonstrate technical feasibility, existing studies often emphasize specific infrastructures or disease context rather than examining broader, generalizable implementation strategies within data platforms. This thesis addresses that gap by investigating two possible approaches for integrating \ac{ML} within XNAT, with the goal of identifying best practices, recognizing limitations, and proposing future measures to support scalable and generalizable AI implementations in healthcare data platforms.

First, this bachelor thesis will examine the Docker Container approach within XNAT, considering its benefits, disadvantages, and future development. It will assess its relevance to the Somnolink project and discuss its applicability regarding sustainability and the FAIR principles. The findings will be synthesized into a clear User Guide for the practical application of the Docker Container approach.
Second, the study will analyze the XNAT Pipeline approach and compare it to the  assessment of the Docker Container method, exploring future measurement possibilities for both.


Chapter 2 lays the fundamentals by explaining the primary concepts necessary for understanding the thesis. Chapter 3 then details the Docker Container method, explaining its deployment within XNAT and introducing the automation script. A user guide is provided for users. Following this, the thesis addresses the Pipeline method and its deployment within XNAT. Chapter 4 provides a comprehensive analysis of both the Docker Container and Pipeline methods within XNAT. It examines their generalizability, relevance to the Somnolink project including sustainability and FAIR criteria, and their advantages and disadvantages. Furthermore, the chapter explores potential future measurements to further assess both methods.



 


  
\chapter{Fundamentals}

An overview of the system and methods used is provided in this chapter, serving as a foundation for the more detailed examination of the approaches that follows. 

\section{The Extensible Neuroimaging Archive Toolkit}
The Extensible Neuroimaging Archive Toolkit (XNAT)~\cite{marcus_extensible_2007} is an open-source imaging informatics platform developed by the Neuroinformatics Research Group at Washington University. It facilitates common management, productivity, and quality assurance tasks for imaging and associated data. Thanks to its extensibility, XNAT can be used to support a wide range of imaging-based projects.\footnote{\url{https://www.xnat.org/about/} (accessed on 10 August 2025)} Due to the fact that XNAT offers the opportunity to use the power of high-performance computing on your data~\cite{zaschke_extending_2024}, 
pipeline processing is one of the options that can be done on XNAT, as well as the deployment of Docker containers.
\\
Additionally, XNAT has a structured website that is composed of a Dashboard that gives an overview of the site, a Project list that browses or searches available projects, and a Project page where all subjects and their data in this project are located. Along with Subject/Session pages where all details are given, including imaging sessions and scan-level details, as well as the section of scans where all the patient’s scans are found. 
This structure of XNAT provides the information on where the patient files are stored, including the composed URL for REST API use. A REST API is a type of application programming interface that adheres to the principles of the REST architecture. REST, which stands for Representational State Transfer, defines a set of conventions for designing web APIs in a consistent and scalable way.\footnote{\url{https://www.redhat.com/en/topics/api/what-is-a-rest-api} (accessed on 10 August 2025)}
\\
\section{The Container}
A container is a standard unit of software that packages up code and all its dependencies so the application runs quickly and reliably from one computing environment to another.\footnote{\url{https://www.docker.com/resources/what-container/} (accessed on 10 August 2025)}
Docker is a platform that allows certain applications to run in software containers. It was launched in 2013.\footnote{\url{https://fr.wikipedia.org/wiki/Docker\_(logiciel)} (accessed on 10 August 2025)} A Docker container allows the user to deploy any application in any environment without worrying about incompatibility issues, regardless of the machine’s configuration settings.\footnote{\url{https://sematext.com/glossary/docker/} (accessed on 10 August 2025)} Although there are some associated terms that will always be needed when it comes to using or learning about Docker containers.
One of the most important of them is the Dockerfile. The Dockerfile is a file that contains the build instructions for the image that the Docker Engine (an open-source containerization technology for building and containerizing your applications\footnote{\url{https://docs.docker.com/engine/} (accessed on 10 August 2025)}) will run. Conversely, there are numerous commands that must be written in the Dockerfile in order to create the image. After building the image, we can tag and push the image to Docker Hub. For this purpose, we just have to run some basic simple commands on a terminal.
\\
Another element to consider is the REST APIs. A REST API is an application programming interface (API) that follows the design principles of the REST architectural style. REST is short for Representational State Transfer and is a set of rules and guidelines about how you should build a web API.\footnote{\url{https://www.redhat.com/en/topics/api/what-is-a-rest-api} (accessed on 10 August 2025)} Luckily, XNAT provides its users with a list of REST APIs that make the exchange between the user and the website possible. 

\section{The JSON Command}
At the same time, if we want XNAT to execute my created Docker image and run a command, a \ac{JSON} command is needed. The JSON command is a collection of properties that describe the Docker image, which allows XNAT to understand what the command is about. In other words, the JSON command is a sort of configuration for XNAT. It answers the following questions for XNAT: What kind of image is it? Does it have a human-friendly name we can use for it? What does the command-line string look like? Does it need files? Where should they go? How do you want to get those out of XNAT? Does it produce any files at the end? Those have to get back into XNAT, so where should they go?\footnote{\url{https://wiki.xnat.org/container-service/json-command-definition} (accessed on 10 August 2025)} These are the fundamental knowledge elements needed in order to achieve automation of the process.

\section{The Pipeline}
Pipelines in XNAT are efficient lightweight applications that can be run on the project data to facilitate sophisticated processing tasks or leverage the computational power of extensive computing clusters. Some pipeline-enabled workflows run automatically, such as auto-QC of images, and some of them require manual steps, like defining a region of interest. After integrating the pipeline in project data, we can set up project-based workflows with project-specific and experiment-specific parameters, track a pipeline and send email notifications, and capture provenance information as the pipeline executes.\footnote{https://wiki.xnat.org/documentation/running-pipelines-in-xnat {last accessed 07.09.2025}} The pipelines are used for tasks like converting DICOM to NIFTI, performing image quality assessment (QA), correcting MRI bias fields, and running advanced analyses such as the FreeSurfer pipeline.\footnote{https://github.com/jhuguetn/xnat-pipelines {last accessed 07.09.2025}}

\section{The Pipeline Descriptor}

Pipelines are identified by using a pipeline descriptor and resource descriptors. A resource descriptor describes an executable. An executable is identified by its name, its location, and the arguments that it takes.\footnote{https://wiki.xnat.org/documentation/xnat-pipeline-development-schema {last accessed 07.09.2025}} In other words, the Pipeline Descriptor is a recipe of the pipeline that tells XNAT exactly what a pipeline is, what input is needed, and where the output should be located, ensuring a seamless integration in XNAT.









  \chapter{Deployment of Methods in XNAT}
\section{The Docker Method}
This chapter provides an analysis and an explanation of the automation script.

\subsection{Introduction to the Automation Script for the Docker Container Method}
Generally these are the following steps to deploy a Docker Container in XNAT manually: writings a external script and building the image and pushing it to Docker Hub, following with writing a appropriate JSON command and deploying it in XNAT server then enabling the command side wide and project wide then finally launching the Docker Container. 
After reviewing the latest updates supported by the XNAT Community, it was confirmed that automating Docker Container deployment in XNAT is feasible.
XNAT's REST API provides functionality to list projects, subjects within those projects, sessions, and scans. It also allows the extraction of patient files, as well as the retrieval of the command \ac{ID} and wrapper \ac{ID}.
To summarize all the REST APIs used in the automation script, Table A.1 in the Appendix provides a comprehensive overview of each API and its purpose.

\begin{figure}[H]
    \centering
    \def\svgwidth{0.4\linewidth}
    \input{RESTAPI.pdf_tex}
    \caption{ Diagram: Illustrates the role of the REST API in automation.}
    \label{fig:RESTAPI}
\end{figure}


The \autoref{fig:RESTAPI} illustrates the role of the REST API in the automatization.



\subsection{Architecture and Organization of the Code}
The automation script mirrors the manual workflow in 14 steps. Each part of the script automates a specific step of the manual process.  
The Python Requests~\cite{request} library is the go-to solution for making \ac{HTTP} requests in Python. The Sys module~\cite{sys} in Python provides access to variables and functions that interact closely with the Python interpreter and runtime environment.
Since a JSON-based command is written, the use of the JSON library~\cite{pythonjson} was necessary. Using the Python Subprocess~\cite{subprocess} module resembles executing commands directly on the system computer using Python. The os enables the use of operating system dependent functionality~\cite{os}.   
The getpass module provides a secure way to handle password prompts where programs interact with the users via the terminal~\cite{getpass}. 

The urllib3~\cite{urllib3} library was used to disable the Insecure Request Warning, which is normally shown when Requests connects to a site with an unverified SSL certificate. The \autoref{fig:diagram-core-libraries} illustrates all the libraries used in the script.
The system runs Docker Engine 28.3.0, using Docker Client 20.10.5+dfsg1 for management. The XNAT web application is running version 1.9.1.1.

\begin{figure}[H]
  \centering
  \def\svgwidth{0.4\linewidth}
  \input{lib.pdf_tex}
  \caption{Schema: Core Libraries for Python Scripting in System Integration.}
  \label{fig:diagram-core-libraries}
\end{figure}
 
  \subsection{The Dockerfile}
 
The first function responsible for writing the Dockerfile is called \texttt{def write\_dockerfile}, and it takes three arguments: the target directory for the Dockerfile, the name of the Python script to include, and an optional base Docker image (defaulting to \texttt{python:3.10-slim}~\cite{Dockerbaseimage}). The function creates the necessary folder (if it does not already exist), constructs the Dockerfile content, and writes it. The content of the Dockerfile is introduced with string interpolation and contains the base image and several commands such as \texttt{WORKDIR /app} (a working directory) and \texttt{COPY \{script\_filename\} /app/\{script\_filename\}} to copy the user script inside the image. The line \texttt{RUN pip install --no-cache-dir pandas} tells Docker to install the pandas Python library using pip during the build process of the Docker image, which helps reduce the overall image size and avoids unnecessary layers in the build process. The last command of the Dockerfile is \texttt{COPY requirements.txt /app/requirements.txt}, which provides the opportunity for the user to write in a separate text file all the libraries used in the container script, to be installed later during the image build process.



\lstdefinestyle{allblack}{
  basicstyle=\ttfamily\small\color{black},
  keywordstyle=\color{black},
  commentstyle=\color{black},
  stringstyle=\color{black},
  identifierstyle=\color{black},
  numberstyle=\color{black},
  rulecolor=\color{black},
  showstringspaces=false
}

\lstset{style=allblack}


\normalsize
\lstset{inputpath=en/content}

\lstinputlisting[
  inputencoding=cp1252, 
  language=Python,
  firstline=18,
  lastline=24,
  caption={Extracted from \texttt{automat\_f\_2.py} (lines 18--24)},
  label={lst:automat-snippet}
]{automat_f_2.py}

\noindent\footnotesize{ See the corresponding lines on GitHub:\url{ https://github.com/tanaebousfiha/XNAT-implemetation/blob/e3c1a547dc8241c4a24f651549069ee042c62393/Automation%20of%20the%20manuall%20Process%20of%20the%20xnat%20implementation/automat_f_2.py#L18-L24}}

\normalsize


The rest of the function creates and writes the Dockerfile to the specified directory, confirming successful completion.




\lstinputlisting[
  inputencoding=cp1252,  language=Python,
  firstline=25,
  lastline=29,
  caption={Extracted from \texttt{automat\_f\_2.py} (lines 25--29)},
  label={lst:automat-snippet}
]{automat_f_2.py}

\noindent\footnotesize See the corresponding lines on GitHub:\url{ https://github.com/tanaebousfiha/XNAT-implemetation/blob/e3c1a547dc8241c4a24f651549069ee042c62393/Automation%20of%20the%20manuall%20Process%20of%20the%20xnat%20implementation/automat_f_2.py#L25-L29}
\normalsize






The function creates valid Dockerfiles and manages external dependencies to prevent errors.

\subsection{Building, Pushing and Tagging the Image}

XNAT requires Docker Hub for container deployment. Users must have Docker Hub account. 

\lstinputlisting[
  inputencoding=cp1252,  language=Python,
  firstline=32,
  lastline=49,
  caption={Extracted from \texttt{automat\_f\_2.py} (lines 32--49)},
  label={lst:automat-snippet}
]{automat_f_2.py}

\noindent\footnotesize See the corresponding lines on GitHub:\url{https://github.com/tanaebousfiha/XNAT-implemetation/blob/3a565f4b666a5f2938d5a41002049ee87cd33b61/Automation%20of%20the%20manuall%20Process%20of%20the%20xnat%20implementation/automat_f_2.py#L32-L49}
\normalsize
\\
The function in the Listing 3.3 builds and pushes a Docker image to Docker Hub. It first prompts the user to enter their Docker Hub username and, if provided, constructs a full image tag in the format <username>/<image\_name>. it uses the \texttt{subprocess} module to run a Docker build command with the specified Dockerfile, capturing any output or errors. If the build fails, the script exits with an error message, otherwise, it proceeds to push the built image to Docker Hub using another docker push command. After a successful push, the function prints a confirmation message and returns the full image tag.



\subsection{The Prompt Function for the Required Input}

This function captures various user inputs required by the script. These include the Docker Hub username, command details for JSON configuration, and login credentials.
The function looks like this:
 
\lstinputlisting[
  inputencoding=cp1252,  language=Python,
  firstline=59,
  lastline=65,
  caption={Extracted from \texttt{automat\_f\_2.py} (lines 59--65)},
  label={lst:automat-snippet}
]{automat_f_2.py}

\noindent\footnotesize See the corresponding lines on GitHub:\url{ https://github.com/tanaebousfiha/XNAT-implemetation/blob/e3c1a547dc8241c4a24f651549069ee042c62393/Automation%20of%20the%20manuall%20Process%20of%20the%20xnat%20implementation/automat_f_2.py#L59-L65}
\normalsize

The \texttt{def get\_input} function uses a \texttt{while True} loop to keep asking until the user enters input.

\subsection{The Command Definition in JSON Format}

To communicate the Docker image to XNAT, it is necessary to write a command definition in JSON format~\cite{JSONCommand}.
To achieve that, the function used is called \texttt{create\_json\_file}, which builds the configuration dictionary for a Docker-based XNAT command (for the XNAT container Service), writes it to a \texttt{command.json} file, and returns the filename.
Let us start analyzing the first block of the JSON command:



\lstinputlisting[
  inputencoding=cp1252,  language=Python,
  firstline=120,
  lastline=131,
  caption={Extracted from \texttt{automat\_f\_2.py} (lines 120--131)},
  label={lst:automat-snippet}
]{automat_f_2.py}

\noindent\footnotesize See the corresponding lines on GitHub:\url{ https://github.com/tanaebousfiha/XNAT-implemetation/blob/e3c1a547dc8241c4a24f651549069ee042c62393/Automation%20of%20the%20manuall%20Process%20of%20the%20xnat%20implementation/automat_f_2.py#L120-L131}
\normalsize


In this block, we are defining the name of the command, adding a description, and specifying the version and the type of the image.
The command line declares the actual command to run inside the container when invoked by XNAT. The placeholder \# is a template substitution (not a regex expression) that tells XNAT: \texttt{"When you launch the Docker command, replace \#input\_file\# with the actual file name/path that the user selected as input."} It does not define what a valid file looks like; instead, it marks a spot in the command where XNAT should "fill in" an actual value.


The mount configures the directory mappings (as Docker volumes) between XNAT and the Docker container. Most of the time, two are used: the input and the output.
We specified the path \texttt{/input} in the container and indicated whether the container can only read the files (\texttt{Not writable}) or can also write them (\texttt{Writable}).

The second part of command definition in JSON format:


\lstinputlisting[
  inputencoding=cp1252,  language=Python,
  firstline=132,
  lastline=160,
  caption={Extracted from \texttt{automat\_f\_2.py} (lines 132--150)},
  label={lst:automat-snippet}
]{automat_f_2.py}

\noindent\footnotesize See the corresponding lines on GitHub: \url{ https://github.com/tanaebousfiha/XNAT-implemetation/blob/e3c1a547dc8241c4a24f651549069ee042c62393/Automation%20of%20the%20manuall%20Process%20of%20the%20xnat%20implementation/automat_f_2.py#L132-L150}
\normalsize

This block defines the parameters that a user must provide as an input file when launching the container.
Let’s break down each field:

The \texttt{"name": "input\_files"} is used in other parts of the JSON as a reference such as placeholders in the command line. In addition, we add an optional human-friendly description. The \texttt{"type"} indicates that the input accepts more than just text files, scans, or numbers; it specifies that each element in the input is specifically a file through the \texttt{"element"} field.

The \texttt{"required": true} flag means that if the input is not provided, the command cannot be run. The mount in this part links the input to a specific mount inside the container. All files provided by the user will appear in the \texttt{/input} directory inside the container. With \texttt{"multiple": true}, it is indicated that the user can upload or select more than one file for this input.

The same logic applies to the output part. The only additional point to note is the \texttt{"path": "."}, which means the output file will be placed at the top level of the \texttt{/output} directory. To allow more than one output file, the setting \texttt{"type": "files"} is used in the output block.

The final block in the command definition (JSON format):


\lstinputlisting[
  inputencoding=cp1252,  language=Python,
  firstline=152,
  lastline=173,
  caption={Extracted from \texttt{automat\_f\_2.py} (lines 152--173)},
  label={lst:automat-snippet}
]{automat_f_2.py}

\noindent\footnotesize See the corresponding lines on GitHub:\url{ https://github.com/tanaebousfiha/XNAT-implemetation/blob/c95b34d9d5a4b236f9b482e02258094384623892/Automation%20of%20the%20manuall%20Process%20of%20the%20xnat%20implementation/automat_f_2.py#L152-L173}
\normalsize



This block configures the command's appearance, permissions, data connections, and input handling within XNAT.

The wrapper name is the technical name of the command inside XNAT. The \texttt{"label": mod\_data["label\_name"]} is a human-readable name shown in the XNAT \ac{UI}. The \texttt{"description": mod\_data["label\_description"]} is a description for the tool/wrapper, shown when users browse tools in XNAT. The \texttt{"context"} specifies where we want the container to appear in the XNAT interface. Since the main idea was to have one container on top of the structure of XNAT, here \texttt{["xnat:projectData"]} means this command is available at the project level.


In the \texttt{"external-input"} block, external entities from XNAT are defined. Because we used the project as the context for the command, we require in the external input that a project be selected. In more detail, the name and type must be "Project." This input must be provided, which is why \texttt{"required": true} is specified. We tell XNAT with \texttt{"load-children": true} to load (in the UI) the child objects of the project when showing this input.

In the \texttt{"output-handlers"} block, we control how the outputs from the command are stored and shown in XNAT after job completion. This output handler tells XNAT to take the result file produced by the command, store it as a resource under the project, and call that section ``Results'' in the UI. The \texttt{"as-a-child-of": "project"} specifies that the results are uploaded back into the XNAT project as new resources.
\normalsize
Finally, the script writes the command definition in JSON format, ensuring that the JSON command is saved in the correct format, ready to be uploaded to XNAT.

\subsection{Upload the Command in XNAT}

After writing the JSON-based command, the next step is to send it to XNAT. To achieve this, we have to use the appropriate REST API responsible for uploading the container command to XNAT.
We can find the list of all REST APIs under \texttt{Administer $\rightarrow$ Site Administration $\rightarrow$ Miscellaneous $\rightarrow$ Development Utilities $\rightarrow$ Swagger}.
To send the command to XNAT, we found that the responsible REST API is \texttt{POST}~\cite{ContainerRESTAPILIST}.

The function responsible for this is \texttt{def send\_json\_to\_xnat}, and it expects four parameters: \texttt{json\_file\_path, xnat\_url, xnat\_user, xnat\_password}. When working with REST APIs, we first have to build the URL endpoint for XNAT’s command registration.
 
\lstinputlisting[
  inputencoding=cp1252,  language=Python,
  firstline=187,
  lastline=200,
  caption={Extracted from \texttt{automat\_f\_2.py} (lines 187--200)},
  label={lst:automat-snippet}
]{automat_f_2.py}

\noindent\footnotesize See the corresponding lines on GitHub:\url{ https://github.com/tanaebousfiha/XNAT-implemetation/blob/c95b34d9d5a4b236f9b482e02258094384623892/Automation%20of%20the%20manuall%20Process%20of%20the%20xnat%20implementation/automat_f_2.py#L187-L200}
\normalsize

This function registers a JSON command with XNAT. It constructs the REST API endpoint, loads the JSON data, sends it via a POST request with HTTP Basic Authentication, and provides feedback on the server's response.

\subsection{Preparations to Launch the Container}

To prepare for launching the container in XNAT, the script automatically retrieves the command ID and wrapper ID from the XNAT web service using REST APIs.
Before launching the container, the script automatically retrieves the command ID and wrapper ID from XNAT using REST APIs.
\lstinputlisting[
  inputencoding=cp1252,  language=Python,
  firstline=204,
  lastline=228,
  caption={Extracted from \texttt{automat\_f\_2.py} (lines 204--228)},
  label={lst:automat-snippet}
]{automat_f_2.py}

\noindent\footnotesize See the corresponding lines on GitHub:\url{ https://github.com/tanaebousfiha/XNAT-implemetation/blob/a47cafecee7f9e467356ddb9f7939371e3d02e03/Automation%20of%20the%20manuall%20Process%20of%20the%20xnat%20implementation/automat_f_2.py#L204-L228}
\normalsize


This function retrieves the command ID and wrapper ID from XNAT, using the hostname, credentials, command name, and wrapper name as inputs. It constructs the REST API endpoint, checks for a successful response, and returns the IDs. If a wrapper name is provided, it searches for a matching wrapper ID. Following ID retrieval, the enable\_wrapper\_sitewide and enable\_wrapper\_for\_project functions enable the command, using a PUT request to the appropriate XNAT REST API endpoint and providing user feedback.

\subsection{The Extraction of All Files from All Project Structures}

The get\_all\_files\_all\_levels function comprehensively collects all files from an XNAT project's data hierarchy (project, subjects, experiments, scans) using the XNAT REST API (GET). It structures the file information into a dictionary containing the file's level, IDs, name, resource folder. The \autoref{fig:extraction} illustrates all the extraction of all the files from a project structure in XNAT.

\begin{figure}[H]
    \centering
    \def\svgwidth{0.7\linewidth}
    \input{extraction.pdf_tex}
    \caption{Diagram: The extraction of files from the entire project structure.}
    \label{fig:extraction}
\end{figure}

\subsection{Launch the Container}

The launch\_container\_with\_all\_files function launches a container with extracted files, using XNAT connection details, project/command identifiers, authentication, wrapper name, and the list of files. It creates a payload, constructs the launch URL, sends a POST request with the payload, and reports the status.


\lstinputlisting[
  inputencoding=cp1252,  language=Python,
  firstline=359,
  lastline=393,
  caption={Extracted from \texttt{automat\_f\_2.py} (lines 359--393)},
  label={lst:automat-snippet}
]{automat_f_2.py}

\noindent\footnotesize See the corresponding lines on GitHub:\url{ https://github.com/tanaebousfiha/XNAT-implemetation/blob/3a565f4b666a5f2938d5a41002049ee87cd33b61/Automation%20of%20the%20manuall%20Process%20of%20the%20xnat%20implementation/automat_f_2.py#L359-L393}
\normalsize


The \autoref{fig:automatic/manuall} illustrates how the automation script facilitates the process of deploying a container in XNAT. It reduces all the steps.

\begin{figure}[H]
    \centering
    \def\svgwidth{0.8\linewidth}
    \input{compare1.pdf_tex}
    \caption{Diagram: The difference between the manual and the automatic process.}
    \label{fig:automatic/manuall}
\end{figure}



\subsection{Testing the Code}
Initially, the script requests user input for login credentials, project ID, command name, and description. Following this input, the script automates image building, tagging, pushing, JSON based command creation, command enabling, and container launching.

\subsection{Results on XNAT}
The result that we came to after testing the code is that all the steps are automated correctly, except that the container could not receive any files. With the \ac{Stdout} view we found out that the container received zero files. The Docker Container output is shown in the \autoref{lst:stdout}.
 
\begin{lstlisting}[numbers=none, caption={stdout in XNAT.}, label={lst:stdout}]
View stdout (from file)
Inhalt von /input (rekursiv):
/input: []
Keine CSV-Datei in /input gefunden.
\end{lstlisting}


\normalsize
But interestingly, after searching in the container information, we noticed that the container in fact received the files and that the files are listed in the input part of the container input. 
Details of the test procedure are described in Appendix 
\autoref{app:Result}. Due to the fact that the list of files was extremely long, it was shortened. In summary, the container received a huge amount of files.
\normalsize
\subsection{Deployment Challenges}
 In this part the deployment challenges are outlined. Initial tests revealed the container wasn't receiving input files, confirmed by a "no\_file.txt" test. Further testing with REST API integration within the Docker Container also failed to produce file results in XNAT. Adjustments to the JSON command, including credentials and detailed output specifications, were unsuccessful. Adding delays between container executions also didn't resolve the input file issue.
Two deployment strategies were tested. Passing all project files to a single Docker container failed due to file transfer issues. Deploying individual containers per project, each processing only local files, was successful, though it resulted in multiple container instances. Chapter 4 provides a detailed discussion of these deployment challenges. 


\normalsize



\subsection{User Guide}
To better understand the system created, a user guide was inserted \autoref{tab:automatic_script} to list how the the workflow is working and what are the requirement that the user needs to have in order to run the script. 

\begin{table}[H]
  \centering
  \caption{ User guide for Automatic Containerization.}
  \label{tab:automatic_script}
  \begin{tabular}{|l|p{9cm}|}
  \hline
  \textbf{System Requirements} & 
  - Python installed. \newline
  -\texttt{requirements.txt} in the same folder as \texttt{automat\_f\_2.py}. \newline
  - XNAT server accessible. \newline
  - Docker Engine installed and running. \newline
  - Logged in to Docker Hub. \\  \hline
  \textbf{Steps to Follow} & 
  1. Run the \texttt{automat\_f\_2.py} script. \newline
  2. Enter the prompted inputs. \newline
  3. Wait until the container is launched. \\
  \hline
  \textbf{What Happens Automatically} & 
  - Dockerfile is generated. \newline
  - Docker image is built, tagged, and pushed. \newline
  - Command JSON is generated. \newline
  - Command is uploaded to XNAT. \newline
  - Wrapper is activated site-wide and project-wide. \newline
  - Container is launched via REST API. \\
  \hline
  \textbf{Problems / Troubleshooting} & 
  $\bullet$ Push failed: check Docker Hub login. \newline
  - Wrapper not found or JSON upload failed: check chosen command naming. \newline
  $\bullet$ Command already exists: delete old command manually. \newline
  $\bullet$ Container problem uploading files: check if the transfer was successful by running \texttt{ls -al \*} on your output directory and check the container \texttt{stdout} log. \\
  \hline
  \end{tabular}
\end{table}


\normalsize

  
\section{Pipeline Method}

Integrating a pipeline in XNAT requires an Extensible Markup Language (XML) descriptor to connect the XNAT with the defined external script. As in the fundamentals cited the XML descriptor defines the Pipeline execution steps among them the input, the output. In this work, the pipeline approach was analyzed using an OSA (Obstructive Sleep Apnea) prediction script as a use case.
\normalsize

\lstinputlisting[
  inputencoding=cp1252,  language=Python,
  firstline=1,
  lastline=31,
  caption={Extracted from \texttt{osa\_pipeline.xml} (lines 1--31)},
  label={lst:automat-snippet}
]{osa_pipeline.xml}

\noindent\footnotesize See the corresponding lines on GitHub:\url{ https://github.com/tanaebousfiha/XNAT-implemetation/blob/f1ea7a23800a9a1dbd2ca8f5a79685f9ab3f04bb/XNAT%20ML%20Plugin/osa_pipeline.xml#L1-L31}
\normalsize
\\
The \autoref{lst:automat-snippet} presents a resource descriptor for a Python script, specifying its location, command prefix, and input arguments such as configuration file. First, which defines the workflow, is copied to /data/xnat/pipeline/pipelines/my\_pipeline/, where XNAT stores pipeline descriptors. 

The XNAT pipeline system consist of two main components: \\
the Pipeline Engine, which executes workflow definitions, and the XNAT Pipeline Engine Plugin, which enables interaction with the XNAT interface.
By default, pipelines are stored in the directory /data/xnat/home/pipeline/. During the initial setup, the Pipeline Engine is initialized via a Secure Shell (SSH) session by running
<PIPELINE\_HOME>/setup.sh YOUR\_ADMIN\_EMAIL\_ID YOUR\_SMTP\_SERVER,
which generates the configuration file pipeline.config. This file contains essential parameters such as the site identifier and administrator email address\footnote{\url{https://wiki.xnat.org/documentation/installing-pipelines-in-xnat} last accessed 10.09.2025}.\\
Usually a complete Pipeline in XNAT is composed of three XML configuration layers:\\
First a resource descriptor: which defines the executable component. \\
Second parameters File, which provides concrete runtime values for these input arguments, such as configuration files.\\
Third pipeline Descriptor, which  specifies which scripts to execute, in which order, and how the outputs should be handled \footnote{\url{https://groups.google.com/g/xnat_discussion/c/ewhjL7upJf4/m/0Soz5-sBEzwJ} last accessed 06.10.2025}. The \autoref{fig:pipeline} defines the essentials element to run a pipeline in XNAT. 
\begin{figure}[H]
    \centering
    \def\svgwidth{0.7\linewidth}
    \input{pipeline.pdf_tex}
    \caption{ Diagram: Conceptual diagram of the XNAT Pipeline Engine workflow.}
    \label{fig:pipeline}
\end{figure}

\subsection{Pipeline Workflow in XNAT}
When a pipeline is launched, first XNAT reads the pipeline Descriptor XML, identifies the recourse descriptor, substitutes the variables defined in the Parameters File, and then executes the specified commands. The outputs are automatically captured and stored as XNAT resources.
Unlike the Docker-based approach, setting up a pipeline in XNAT requires more manual steps. First, the pipeline path must be registered under Admin > Pipelines > Register new Pipelines. Then, to allow users to run the pipeline from the XNAT interface, an administrator must edit the relevant data type under Administer > Data Types and add a new action (PipelineScreen\_launch\_pipeline) in the available report actions section, with a display label such as Build or Run Pipeline\footnote{\url{https://wiki.xnat.org/documentation/installing-pipelines-in-xnat} last accessed 10.09.2025}.
\normalsize



\subsection{The Pipeline and the REST API}
With the aim of analyzing the Pipeline method more deeply, it is worth noting that XNAT provides several REST API endpoints for managing and running pipelines the most relevant endpoint for automation is \url{POST /pipelines/launch/{pipelineName}} \footnote{\url{://wiki.xnat.org/xnat-api/xnat-pipeline-api} last accessed 10.09.2025}.
This enables programmatic execution of registered pipelines in XNAT. However, full automation is limited, as endpoints to globally enable a pipeline or assign it to a specific project are missing from the REST API. However, while pipeline execution can be automated, deployment and project-level setup still require manual steps.

\section{User Guide for the Pipeline Approach}
To gain a better understanding of the pipeline approach, \autoref{tab:pipeline} provides an explanation of all the steps involved in this approach.
\begin{table}[H]
  \centering
  \caption{ User guide the pipeline method.}
  \label{tab:pipeline}
  \begin{tabular}{|l|p{9cm}|}
  \hline
  \textbf{Download Requirements steps} & 
  1. Pipeline Engine installed\footnote{\url{https://wiki.xnat.org/documentation/installing-the-pipeline-engine} last accessed 11.10.2025}. \newline
  2. Pipeline Engine Plugin installed\footnote{\url{https://wiki.xnat.org/xnat-tools/xnat-pipeline-engine-plugin} last accessed 11.10.2025} \newline
  3.Create and configure the pipeline descriptor component. \newline
  4.Make sure the resource descriptor, parameter file, and pipeline descriptor are located in the same directory under the Pipeline Home path: \texttt{/data/xnat/pipeline/pipelines/my\_pipeline/} \newline
  5.Ensure Secure Shell (SSH) access is available and the pipeline script has execution permission using: chmod +x my\_pipeline.xml. 
   \\  \hline
  \textbf{Enabling the Pipeline in XNAT} & 
    1. In the XNAT interface, go to Administer → Pipelines. Provide the full path to your pipeline directory. \newline
    2. Then navigate to Administer → Data Types.\newline
    3. Click on the desired data type element \newline
    4. Click Edit in the top-right corner of the modal window.\newline
    5. Under Available Report Actions, add the following values:\newline
    - Name: PipelineScreen\_launch\_pipeline\newline
    -Display Name: Build\newline
    -Popup: always\newline
    -Secure Access: edit\newline
  \\ \hline
  \textbf{Launching the Pipeline} & 
   - Manually: From the data type page, click Run/Build Pipeline. \newline
   - Programmatically via REST API\footnote{\url{https://wiki.xnat.org/xnat-api/xnat-pipeline-api} last accessed 12.10.2025}.\newline
   -Once the pipeline starts, you will receive an notification confirming the launch or completion\footnote{\url{https://wiki.xnat.org/documentation/running-pipelines-in-xnat} last accessed 12.10.2025}. 
   \\
  \hline
  \textbf{Problems / Troubleshooting} & 
  - Pipeline not found:\newline
    Ensure all pipeline descriptor files are correctly placed in the Pipeline Home directory. \newline
  - Pipeline failed execution:
   Re-run the following command to reset file permissions:\newline
    chmod +x my\_pipeline.xml
  - Check logs: Review the XNAT logs for details\newline
  - Incorrect pipeline registration: Verify that the pipeline path is correctly configured under Administer → Pipelines.
  
  \\
  \hline
  \end{tabular}
\end{table}
  \chapter{Comprehensive Analysis}
This chapter conducts the comprehensive analysis for both methods Docker Container and Pipeline. 
A full comparison with and disadvantages and advantages for both methods are presented.
\section{Docker Method}
\subsection{The Docker Container workflow in XNAT}

XNAT provides data flow for manually run Docker containers. It generates a JSON specification, prepares and mounts data, and reloads the container's results.
The diagram below illustrates the workflow data for a clearer understanding.  
The \autoref{fig:workflow} explains the workflow data in XNAT.

\begin{figure}[H]
    \centering
    \def\svgwidth{\linewidth} 
    \input{xnat.pdf_tex}
    \caption{Diagram: XNAT Workflow for Container Integration.}
    \label{fig:workflow}
\end{figure}

\subsection{Generalizability of the Docker method}
The Docker Container method is a suitable way to integrate AI into XNAT, working well with multiple data types and its recommended from the XNAT Documentation to launch jobs with the Docker Container method. Complete workflow automation was achieved with the Docker Container method, enabling users to select and transfer specific files to the container at various project data levels. However, the REST API call failed to transfer all project-level files to a single container, despite the files being listed in the container information.  

A senior XNAT developer confirmed that containers lack direct database access by default, requiring credential passing (or a secret store like Vault or Keycloak)~\cite{database}. However, this approach is strongly discouraged due to potential security risks and the possibility of causing harm to the database. Direct container-database connection is unlikely due to PostgreSQL's IP restrictions, which would break containers in Docker Swarm/Kubernetes and explains why the container couldn't receive files.
While the senior developer confirmed that retrieving files via REST API during container execution is a valid approach, no files were successfully transferred or mounted to the container in practice. 
XNAT was able to mount files from the same context but not able to mount the files from other contexts, what explains why when i proceeded files from the same context as the recourse the container could process them. 



To improve generalizability and accessibility especially for non-programming users the method could be extended with a graphical user interface (GUI), making it easier to apply across different institutions, datasets, and environments. 
\normalsize
\subsubsection{GUI Mockup for Docker Container automation in XNAT}
\normalsize
To improve accessibility and usability for non-technical users, the Docker Container  automation script was visualized as a GUI mockup. As shown in the \autoref{fig:mockup}, this mockup demonstrates how the command-line workflow appears, following the same logical steps: (1) Login credentials, (2) Python script path, (3) Description of the command, (4) Project ID, and (5) Launching the container.
\normalsize
\begin{figure}[H]
    \centering
    \def\svgwidth{\linewidth} 
    \input{mockup.pdf_tex}
    \caption{ Mockup Example for the Automation Script.}
    \label{fig:mockup}
\end{figure}
\normalsize
\subsection{The Docker Container approach and the Somnolink project}

The Somnolink project aims to improve obstructive sleep apnea (OSA) diagnose and  treatment and research by enhancing data sharing nationwide~\cite{krefting_somnolink_2025}. The primary goal of the project is to improve the diagnosis, treatment and adherence to treatment of patients with obstructive sleep apnea by using medical informatics and medical data science methods~\cite{aimsomnilink}.
In order to achieve this goal method of informatics and ML-based decisions support system are being used. 
The data source is chosen based on clinical practice, and the OSA patient pathway follows the four-step diagnostic process : (1) assessment of the pretest probability via anamnesis and questionnaires, (2) physical examination, (3) sleep diagnostics by an outpatient sleep specialist, (4) comprehensive diagnostics in a sleep lab ~\cite{krefting_somnolink_2025}.
The \autoref{fig:somnolink} illustrates the structure of the OSA diagnostic pathway in relation to data collection, and data science and ML support.
\begin{figure}[H]
    \centering
    \def\svgwidth{\linewidth} 
    \input{Somnolink.pdf_tex}
    \caption{Diagram: The Docker Container and the Somnolink project.}
    \label{fig:somnolink}
\end{figure}

The fact that the proposed solution for the Somnolink approaches is the ML-based screening of inpatient for OSA, and ML-based therapy response prediction, ML-based prediction of therapy discontinuation risk, the usage of Docker Containers within the Somnolink Project could facilitate the achievement of the project goals, by allowing model implementation within XNAT. Additionally  automating these containers workflows standardizes the execution of Somnolink predictions, reducing manual effort, ensuring workflow consistency, thereby supporting the project's objectives.\\









\subsection{Sustainability Aspects of the Docker Container Including FAIR Criteria}

According to Wilkinson et al. ~\cite{FAIR}, "The FAIR guiding principles describe distinct considerations for contemporary data publishing environments with respect to supporting both manual and automated deposition, exploration, sharing, and reuse."
The use of the Docker Containers contribute by offering reproducible computational environments, reducing maintenance demands, and facilitating collaboration in XNAT. However, analyzing this method concerning the FAIR criteria reveals the following:\\
$\bullet$ Findable(F1–F4):\\
The Docker images are assigned a globally unique and persistent identifier. Their metadata describes the container’s purpose, version, build date, author, and dependencies. Additionally they are registered or indexed in Docker Hub and can be stored in XNAT as well.\\
$\bullet$ Accessible (A1–A2):\\
The Docker images are also accessible via standardized interfaces and they are retrievable using open and standardized protocols(HTTPS, REST API). At the same time the images could be secured with private access. Even with the output file deleted from XNAT, the preserved metadata (filename, ID, container version, etc.) allows future researchers to verify the result's existence, understand its origin, and regenerate the data by re-running the workflow.\\
$\bullet$ Interoperable (I1–I3): \\
Standardized data formats and XNAT-compatible schemas in container outputs ensure seamless integration with other tools and enhance dataset compatibility.\\
$\bullet$ Reusable (R1–R1.3):\\
While Docker containers include detailed metadata promoting reuse, the image owner retains the right to make the image private, limiting accessibility~\cite{FAIR}.











Another key strength for the developed automation script is its sustainability and long-term usability. The implementation is flexible and allows at the same time to the developers a reuse of the existing framework and modify only components that may need updating. Potential future changes may include updates to XNAT REST API endpoints, the introduction of new authentication methods, or alterations to the JSON-based schema for command and wrapper definitions. Additionally, existing Docker base images could become deprecated, or XNAT project structures might be extended with new contexts or data types. However, the core workflow logic would remain unaffected.

\section{Pipeline method}
\subsection{Pipeline workflow in XNAT}
When a pipeline is launched, first XNAT reads the pipeline Descriptor XML, identifies the recourse descriptor, substitutes the variables defined in the Parameters File, and then executes the specified commands. The outputs are automatically captured and stored as XNAT resources.

\subsection{Generalizability of the pipeline}
\normalsize
To generalize a pipeline and to make it transferable to any user and to any user-case, including those with limited informatics expertise, presents several challenges.
Since the method requires manual transfer of the XML descriptor,  and the available endpoints are limited. 
However, it is possible to design a template for the XML descriptor and create a Python script that collects information from the user and substitutes it into the template. Additionally, a graphical user interface (GUI) could be developed to facilitate inserting the pipeline into XNAT. However, because the REST API offers only limited functionality, such a GUI cannot fully communicate the pipeline to XNAT.



%Template für XML bauen
%\url{https://groups.google.com/g/xnat_discussion/c/nFNfCx4K2SA/m/wj8N6VtVAQAJ}
%\url{https://groups.google.com/g/xnat_discussion/c/-1LlALF1vpQ/m/APPP90ofRzAJ}
%\url{https://groups.google.com/g/xnat_discussion/c/ewhjL7upJf4/m/0Soz5-sBEzwJ}



\subsection{The pipeline approach and the Somnolink project, sustainability aspects including FAIR criteria}
The pipeline approach presents challenges adhering to FAIR principles.
Pipeline descriptors are stored in locally on the XNAT server and are not searchable.
Accessibility is also restricted, as only administrators have the necessary permissions to manually register and configure the XML files.
Additionally, the potential for reuse is also limited, since adapting a pipeline to new contexts necessitates manual transfer of XML files and server access configurations. 
After creating and successfully testing the pipeline, the pipeline owners can secure their pipeline descriptors using file system permissions, so that they can be accessed or transmitted by the owner. However the other users must configure all other steps to integrate the pipeline on their own XNAT.
The practical use of pipelines is limited, as they require direct server access and technical configuration, which makes the pipeline not suitable for developers, system administrators or clinicians.


\section{Evaluating pipeline and Docker approaches in XNAT}

For the purpose of evaluating the two methods, i began by compiling a comparative table to provide a clearer understanding of their respective features.

\subsection{Advantages and disadvantages of the methods}

After comparing both approaches in terms of installation, configuration, automation, and maintenance (see ~\autoref{tab:pipeline-vs-docker}), a summary of their main advantages and disadvantages is provided in ~\autoref{tab:docker_pipeline}.

\begin{table}[htbp]
\centering
\caption{Advantages and disadvantages of the Pipeline method and Docker Container method in XNAT.}
\label{tab:adv-disadv}
\renewcommand{\arraystretch}{4}
\begin{tabular}{|p{1cm}|p{3cm}|p{5cm}|p{4cm}|}
\hline
\textbf{Aspect} & \textbf{Pipeline Method} & \textbf{Docker Container Method} & \textbf{Automatization (Docker method)} \\
\hline
\textbf{Advan-tages} &
\parbox[t]{3cm}{
$\bullet$ Fully compatible with older XNAT versions~\cite{Pipelinecompatible}.\\
$\bullet$ Execution of external applications and shell scripts is possible ~\cite{jansen_extending_2015}.
}
&
\parbox[t]{5cm}{
$\bullet$ Easy installation (Docker Engine). \\
$\bullet$ Minimal configuration via JSON-based command.\\
$\bullet$ Highly reusable across XNAT servers. \\
$\bullet$ Easier updates by building and pushing a new image. \\
$\bullet$ Large community support.\\
$\bullet$ Secure via isolated container environment. \\
$\bullet$ Docker images stored on Docker Hub can be used in different clinics and research institutions.\\
}
&
\parbox[t]{3cm}{
$\bullet$ No configuration needed. \\
$\bullet$ Extremely low knowledge requirement . \\
$\bullet$ All manual steps automated. \\
$\bullet$ The script asks simple questions. \\
$\bullet$ Time saving compared to the manual Docker method. \\
$\bullet$ Images reusable across environments, because they are stored in the Docker Hub. \\
$\bullet$ Images securely stored in Docker Hub and within XNAT.\\
} \\
\hline
\textbf{Disadv-antages} &
\parbox[t]{3cm}{
$\bullet$ Requires manual placement of XML descriptors.\\
$\bullet$ Needs server access for installation and updates.\\
$\bullet$ Requires XML schema knowledge.\\
$\bullet$ Limited automation support and no automation achievable.\\
$\bullet$ The pipeline has to be  removed or disabled after the running~\cite{addingpipeline}.\\
$\bullet$ The pipeline is not a preferred way to launch jobs in XNAT.\\

}
&
\parbox[t]{6cm}{
$\bullet$ Requires Docker knowledge. \\
$\bullet$ Depends on Docker availability \\if Docker is blocked or fails, \\images are not recognized\\ within XNAT. \\
$\bullet$ Requires regular expression\\ knowledge for file filtering. \\
$\bullet$ No access to the database. \\

}
&
\parbox[t]{4cm}{
$\bullet$ Problems launching a single container. \\
$\bullet$ For specific file processing, the JSON command in the script must be adjusted, extended or modified.\\
$\bullet$ Selecting a XNAT level and launching with selected file is the successful model achieved. 
} \\
\hline
\end{tabular}
\label{tab:docker_pipeline}
\end{table}


\section{Measures for the future for both methods}

For the Pipeline method, future development should focus on three key areas: Firstly, establishing a secure pipeline Hub to centralize storage and enhance environmental security. Secondly, integrating XML deployment directly within XNAT would streamline the build process for the "Run Pipeline" option, eliminating the need for server access and simplifying overall usability. Thirdly supporting more REST API endpoints.\\
Future development of the Docker Container method should focus on: Firstly, enabling direct access to the XNAT database to facilitate comprehensive file processing across the entire project. Secondly, enhancing the JSON command to allow re-uploading processed files to various output locations within XNAT, regardless of the initial file selection. This would ensure that processed files are returned to their original locations. And lastly implementing the GUI mockup for the automation.
Additionally it is preferable to create a new REST API endpoint for launching containers, within this one a placement for the \texttt{input files} are given, since i faced a problem with communicating the files to the container with REST API, in order to avoid communicating the files through a payload, and obviously this would not be successful if the mounting is restricted for only the defined context. 

\begin{table}[H]
  \centering
  \caption{Comprehensive comparison of XNAT Pipeline method and Docker Container method.}
  \label{tab:pipeline-vs-docker}

  \begin{tabular}{|>{\centering\arraybackslash}p{2cm}|
                      >{\centering\arraybackslash}p{4cm}|
                      >{\centering\arraybackslash}p{4cm}|
                      >{\centering\arraybackslash}p{4cm}|}
    \hline
    \textbf{Aspect} & \textbf{The Pipeline method} & \textbf{Docker Container method} & \textbf{Docker method with Automatization }\\ \hline
    
    
    Installation & Requires manual placement of XML descriptors, plugin JARs~\cite{installpipeline}, and executables on the server. & Container image uploaded once (via Docker registry) and configured in XNAT. & The automatization script must be launched and the Docker Engine must be installed and started up as well. \\ \hline
    

    
    Configuration & Admin must edit data types, enable actions, and rebuild Pipeline engine, require direct access to the server in order to manually copy the XML descriptor and associated scripts to a specific directory. & Minimal configuration through a JSON command, no engine rebuild needed. & No need for any configuration, the script carries the workflow to the end. \\ \hline
    
    Execution & The Run Pipeline option must be built via actions menu. & Triggered via actions menu or REST API, image runs in isolated environment. & The Docker Container is enabled automatically sitewide and project wide too. \\ \hline

    Deployability and flexibility & Pipelines are highly dependent on the specific XNAT environment and server configuration in which they are deployed. & Docker images can be reused in various XNAT installations and environment. & Automatically deploying within XNAT, after that the images can be reused in different environment. \\ \hline
    
    Maintenance & Requires server access and manual reconfiguration. & Updating requires building and pushing a new image, no direct server changes needed. & Old image can be deleted and then creating new images with updated external script.\\ \hline
    
    Usability & Requires XML schema knowledge and direct server administration. & Requires Docker knowledge, but easier to reuse prebuilt images, through running basic commands on terminal. & Extreme low knowledge requirement (except writing a requirement file and answering the script questions. \\ \hline
    
    
    Security & Credentials needed in XML configuration. & Isolated container environments. & Isolated container environments. \\ \hline
  \end{tabular}
\end{table}

 The table \ref{tab:pipeline-vs-docker} presents a direct comparison between the two methods.

  \chapter{Conclusion}


%Starting
This work brings to light the different AI deployment approaches in XNAT. Amongst them the Docker Container method and the Pipeline method. 
%Result
The Docker Container method proved to be a method with a easy installation process, and needs a minimal configuration through the JSON based command. Additionally it provides a automation deployment possibility through the big REST API support from XNAT community. The fact that images are stored in Docker Hub and can be reused across various XNAT installations suggests that the Docker Container method is a sustainable approach, aligning with FAIR principles.
on the other side of the coin, the Docker Container method is heavily reliant on Docker availability and requires Docker expertise.

The Pipeline method proved to be a method with multiple limitations. Since the method requires manual placement of XML descriptors, as well as the admin must extra configure the data types in order to enable the Pipeline on project wide or side wide. Considering that we have unfortunately a limited documentation and less support from XNAT Community the method remains to be non sustainable and not aligning with FAIR principles.

In comparing these two methods, we conclude that the Docker Container method offers significant advantages over the Pipeline method. It also appears to have greater support within the XNAT community.

% Zoom out >>> introducing Deep leaning idea
This thesis has demonstrated the feasibility of deploying a machine learning component using Docker and Pipeline method within XNAT. 
While such approaches such as the Docker Container method proved to be effective, thus still limited in term of capturing the full complexity of medical data.
Conversely, deep learning methods such as convolutional neural networks for imaging and transformers for clinical text are becoming increasingly prominent in healthcare AI.
By extending the presented container or pipeline based to accommodate these models, future work could be able to create clinically impactful AI workflows in XNAT.






 %--- references ---
  \bibliographystyle{IEEEtran}
  \bibliography{content/references}
  \addcontentsline{toc}{chapter}{\bibname}

  %--- lists ---
  \listoffigures
  \listoftables
  \lstlistoflistings

  \newpage
 \section{Acknowledgments}

I would like to sincerely thank my supervisor, Mr. Philip Zaschke, for his support and consistent feedback throughout the writing of my bachelor's thesis. His meticulous attention to detail has greatly shaped the quality of this work. His contributions extended beyond feedback, as he generously shared his expertise and provided invaluable guidance throughout the entire process.

I also wish to express my appreciation to Prof. Dr. Grothausmann for offering the BaMa seminar, which provided valuable feedback that enhanced the quality of my work.

I extend my sincere gratitude to Mr. Christoph Jensen for his generous sharing of expertise and experience. His insightful feedback on my writing was invaluable to this work.


The following software's were used in this thesis: \\ 
I have used the HAWKI and CHATGPT for correcting grammatical mistakes, shorten long written parts. Valuable ideas was with these software also provided.
The use of Stack Overflow for obtaining valuable answers that supported my learning and use of LaTeX.

To create vector graphics, I initially used draw.io, as it made creating diagrams and schematics much easier. I then exported the diagrams as SVG files. However, since ShareLaTeX did not accept the SVG files exported from draw.io, I downloaded the diagrams as PDFs in Inkscape and obtained the PDF and LaTeX code for better illustration quality.













  %--- appendix ---
  \begin{appendix}
    
    
\appendix
\chapter{Appendix}
\section{Test Protocol and Results}
\label{app:test}

\noindent\textbf{Author:} Tanae Bousfiha  
\noindent\textbf{Date:} August 2025  

\noindent This appendix documents the full terminal output of the automated Docker–XNAT integration script
\texttt{Projekt.py}.  
It includes the complete build, push, and wrapper activation process, as well as the execution log showing the files processed by the container.  
This section serves as a reproducibility record for the workflow described in Section~X.X of the main text.


\lstinputlisting[
  linerange={36-41},
  firstnumber=36,
  caption={Auszug aus \texttt{output.txt}},
]{content/output.txt}
  \end{appendix}

\end{document}

