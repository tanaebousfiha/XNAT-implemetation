
\chapter{Fundamentals}

Before proceeding to examine the automation script in more detail, the systems and approaches used will be explained in this chapter.

\section{The Extensible Neuroimaging Archive Toolkit}
The Extensible Neuroimaging Archive Toolkit (XNAT) is an open-source platform for managing and sharing neuroimaging data. It was originally created at Washington University by the Neuroinformatics Research Group in the Buckner Lab. Its extensible framework supports tasks such as data organization, workflow management, and quality control across a variety of imaging research projects.\footnote{\url{https://www.xnat.org/about/}, accessed 23 August 2025}

Because XNAT offers the opportunity to leverage the power of high-performance computing on your data~\cite{zaschke_extending_2024}, 
pipeline processing is one of the options that can be performed on XNAT, as well as the deployment of Docker containers.
\\
Additionally, XNAT has a structured website that is composed of a Dashboard that gives an overview of the site, a Project list that browses or searches available projects, and a Project page where all subjects and their data in this project are located. Along with Subject/Session pages where all details are given, including imaging sessions and scan-level details, as well as the section of scans where all the patient’s scans are found. 
This structure of XNAT provides information on where patient files are stored, including the composed URL for REST API use. A REST API (Representational State Transfer API) is a type of application programming interface that follows the REST architectural style. REST defines a set of constraints and best practices for designing scalable and consistent web services.\footnote{Red Hat. \textit{What is a REST API?}. Available at: \url{https://www.redhat.com/en/topics/api/what-is-a-rest-api}, accessed 23 August 2025.}

\section{The Container} 
A container is a portable unit of software that packages an application together with all the components it depends on, such as libraries and configuration files. This ensures that the application can run reliably in different computing environments. \footnote{Available at: \texttt{https://www.docker.com/resources/what-container/} (accessed 23 August 2025)}

Docker is a platform that enables applications to run within software containers, and it was launched in 2013.\footnote{\url{https://fr.wikipedia.org/wiki/Docker\_(logiciel)} (accessed on 10 August 2025)} A Docker container allows users to deploy any application in any environment without worrying about incompatibility issues, regardless of the machine’s configuration settings.\footnote{\url{https://sematext.com/glossary/docker/} (accessed on 10 August 2025)} There are some associated terms that will always be encountered when it comes to using or learning about Docker containers.
One of the most important of them is the Dockerfile. A Dockerfile is a text file that contains the build instructions for creating a Docker image. This image can then be executed by the Docker Engine, an open-source containerization platform for building and running applications.\footnote{\url{https://docs.docker.com/engine/}, accessed 10 August 2025}. Conversely, there are numerous commands that must be written in the Dockerfile in order to create the image. After building the image, we can tag and push the image to Docker Hub. For this purpose, we just have to run some simple basic commands on a terminal.
\\
Another important point to consider is that REST APIs correspond to application programming interfaces built in compliance with the Representational State Transfer (REST) principles.
REST specifies a collection of standard rules and conventions for web API implementation.\footnote{\url{https://www.redhat.com/en/topics/api/what-is-a-rest-api} (accessed on 10 August 2025)} Luckily, XNAT provides its users with a list of REST APIs that make the exchange between the user and the website possible. 

\section{Command Definition in JSON format}
At the same time, if we want XNAT to execute a created Docker image and run a command, a \ac{JSON} command definition in JSON format is needed. 

The command definition in JSON format is a structured set of properties that defines the Docker image, which enables XNAT to execute the command correctly.\footnote{\url{https://wiki.xnat.org/container-service/json-command-definition}, last accessed 23.08.2025}

In other words, this command definition in JSON format (written in JSON) can be seen as a configuration for XNAT. When defining a command for XNAT in JSON format, the Docker image and a clear, human-readable name are specified, along with the exact command-line invocation, what input files are required and where they are mounted, how those inputs are selected from XNAT, and what outputs are produced and where they should be uploaded back into XNAT.\footnote{\url{https://wiki.xnat.org/container-service/json-command-definition}, accessed 23 August 2025}
These are the fundamental elements needed to enable automation of the process.


