
\begin{abstract}
This work investigates the existing methods of implementation in \ac{XNAT}(eXtensible Neuroimaging Archive Toolkit), with the aim of identifying limitations and best practices. Based on this analysis, an automated implementation workflow was developed. It was hypothesized that a Container could be implemented in a Project in XNAT and to process all files and upload the result files back to their places. The approach is partly based on using a Python Script that worked with \ac{REST API} and creates a Dockerfile and asks the user for an external Container Script and extract all the files from the open source XNAT. Upon the conclusion of the experience, the Docker Container in XNAT should be in ‘Complete’ turned and the result files should be uploaded. Experimental application of the methods demonstrated that the container could not receive the extracted files via REST API, which led to not feasible for either processing or uploading files. Results revealed a significant correlation between the workflow data of XNAT and the mounting of data in the Docker container. These findings demonstrate that the extraction of all the files from all the levels of XNAT leads to a problem mounting in the XNAT host before arriving to the container. And the fact that the Container has no access to data Bank of XNAT conducts that the effectiveness of the method proved to be context dependent and thus limited.
\end{abstract}

